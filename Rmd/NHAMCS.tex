% Options for packages loaded elsewhere
\PassOptionsToPackage{unicode}{hyperref}
\PassOptionsToPackage{hyphens}{url}
%
\documentclass[
]{article}
\usepackage{amsmath,amssymb}
\usepackage{lmodern}
\usepackage{ifxetex,ifluatex}
\ifnum 0\ifxetex 1\fi\ifluatex 1\fi=0 % if pdftex
  \usepackage[T1]{fontenc}
  \usepackage[utf8]{inputenc}
  \usepackage{textcomp} % provide euro and other symbols
\else % if luatex or xetex
  \usepackage{unicode-math}
  \defaultfontfeatures{Scale=MatchLowercase}
  \defaultfontfeatures[\rmfamily]{Ligatures=TeX,Scale=1}
\fi
% Use upquote if available, for straight quotes in verbatim environments
\IfFileExists{upquote.sty}{\usepackage{upquote}}{}
\IfFileExists{microtype.sty}{% use microtype if available
  \usepackage[]{microtype}
  \UseMicrotypeSet[protrusion]{basicmath} % disable protrusion for tt fonts
}{}
\makeatletter
\@ifundefined{KOMAClassName}{% if non-KOMA class
  \IfFileExists{parskip.sty}{%
    \usepackage{parskip}
  }{% else
    \setlength{\parindent}{0pt}
    \setlength{\parskip}{6pt plus 2pt minus 1pt}}
}{% if KOMA class
  \KOMAoptions{parskip=half}}
\makeatother
\usepackage{xcolor}
\IfFileExists{xurl.sty}{\usepackage{xurl}}{} % add URL line breaks if available
\IfFileExists{bookmark.sty}{\usepackage{bookmark}}{\usepackage{hyperref}}
\hypersetup{
  pdftitle={Materials and Methods},
  pdfauthor={Orestis Loukas \& Ho-Ryun Chung},
  hidelinks,
  pdfcreator={LaTeX via pandoc}}
\urlstyle{same} % disable monospaced font for URLs
\usepackage[margin=1in]{geometry}
\usepackage{color}
\usepackage{fancyvrb}
\newcommand{\VerbBar}{|}
\newcommand{\VERB}{\Verb[commandchars=\\\{\}]}
\DefineVerbatimEnvironment{Highlighting}{Verbatim}{commandchars=\\\{\}}
% Add ',fontsize=\small' for more characters per line
\usepackage{framed}
\definecolor{shadecolor}{RGB}{248,248,248}
\newenvironment{Shaded}{\begin{snugshade}}{\end{snugshade}}
\newcommand{\AlertTok}[1]{\textcolor[rgb]{0.94,0.16,0.16}{#1}}
\newcommand{\AnnotationTok}[1]{\textcolor[rgb]{0.56,0.35,0.01}{\textbf{\textit{#1}}}}
\newcommand{\AttributeTok}[1]{\textcolor[rgb]{0.77,0.63,0.00}{#1}}
\newcommand{\BaseNTok}[1]{\textcolor[rgb]{0.00,0.00,0.81}{#1}}
\newcommand{\BuiltInTok}[1]{#1}
\newcommand{\CharTok}[1]{\textcolor[rgb]{0.31,0.60,0.02}{#1}}
\newcommand{\CommentTok}[1]{\textcolor[rgb]{0.56,0.35,0.01}{\textit{#1}}}
\newcommand{\CommentVarTok}[1]{\textcolor[rgb]{0.56,0.35,0.01}{\textbf{\textit{#1}}}}
\newcommand{\ConstantTok}[1]{\textcolor[rgb]{0.00,0.00,0.00}{#1}}
\newcommand{\ControlFlowTok}[1]{\textcolor[rgb]{0.13,0.29,0.53}{\textbf{#1}}}
\newcommand{\DataTypeTok}[1]{\textcolor[rgb]{0.13,0.29,0.53}{#1}}
\newcommand{\DecValTok}[1]{\textcolor[rgb]{0.00,0.00,0.81}{#1}}
\newcommand{\DocumentationTok}[1]{\textcolor[rgb]{0.56,0.35,0.01}{\textbf{\textit{#1}}}}
\newcommand{\ErrorTok}[1]{\textcolor[rgb]{0.64,0.00,0.00}{\textbf{#1}}}
\newcommand{\ExtensionTok}[1]{#1}
\newcommand{\FloatTok}[1]{\textcolor[rgb]{0.00,0.00,0.81}{#1}}
\newcommand{\FunctionTok}[1]{\textcolor[rgb]{0.00,0.00,0.00}{#1}}
\newcommand{\ImportTok}[1]{#1}
\newcommand{\InformationTok}[1]{\textcolor[rgb]{0.56,0.35,0.01}{\textbf{\textit{#1}}}}
\newcommand{\KeywordTok}[1]{\textcolor[rgb]{0.13,0.29,0.53}{\textbf{#1}}}
\newcommand{\NormalTok}[1]{#1}
\newcommand{\OperatorTok}[1]{\textcolor[rgb]{0.81,0.36,0.00}{\textbf{#1}}}
\newcommand{\OtherTok}[1]{\textcolor[rgb]{0.56,0.35,0.01}{#1}}
\newcommand{\PreprocessorTok}[1]{\textcolor[rgb]{0.56,0.35,0.01}{\textit{#1}}}
\newcommand{\RegionMarkerTok}[1]{#1}
\newcommand{\SpecialCharTok}[1]{\textcolor[rgb]{0.00,0.00,0.00}{#1}}
\newcommand{\SpecialStringTok}[1]{\textcolor[rgb]{0.31,0.60,0.02}{#1}}
\newcommand{\StringTok}[1]{\textcolor[rgb]{0.31,0.60,0.02}{#1}}
\newcommand{\VariableTok}[1]{\textcolor[rgb]{0.00,0.00,0.00}{#1}}
\newcommand{\VerbatimStringTok}[1]{\textcolor[rgb]{0.31,0.60,0.02}{#1}}
\newcommand{\WarningTok}[1]{\textcolor[rgb]{0.56,0.35,0.01}{\textbf{\textit{#1}}}}
\usepackage{graphicx}
\makeatletter
\def\maxwidth{\ifdim\Gin@nat@width>\linewidth\linewidth\else\Gin@nat@width\fi}
\def\maxheight{\ifdim\Gin@nat@height>\textheight\textheight\else\Gin@nat@height\fi}
\makeatother
% Scale images if necessary, so that they will not overflow the page
% margins by default, and it is still possible to overwrite the defaults
% using explicit options in \includegraphics[width, height, ...]{}
\setkeys{Gin}{width=\maxwidth,height=\maxheight,keepaspectratio}
% Set default figure placement to htbp
\makeatletter
\def\fps@figure{htbp}
\makeatother
\setlength{\emergencystretch}{3em} % prevent overfull lines
\providecommand{\tightlist}{%
  \setlength{\itemsep}{0pt}\setlength{\parskip}{0pt}}
\setcounter{secnumdepth}{-\maxdimen} % remove section numbering
\usepackage{mathtools}
\newcommand{\abs}[1]{\left\vert#1\right\vert} % absolute value |x|
\newcommand{\prob}[1]{\mathfrak{#1}}
\newcommand{\multidistro}{\text{mult}}

\newcommand{\dimstatespace}{{\abs{\mathcal{A}}}}
\newcommand{\maxent}{\textsc{m}ax\textsc{e}nt~}
\newcommand{\maxentP}{\hat{\prob p}}
\newcommand{\ipf}{\textsc{ipf}~}

\DeclarePairedDelimiterX\infdivx[2]{(}{)}{
  #1\;\delimsize\|\;#2
}
\newcommand{\infdiv}{D_\textsc{kl}\infdivx}
\ifluatex
  \usepackage{selnolig}  % disable illegal ligatures
\fi

\title{Materials and Methods}
\author{Orestis Loukas \& Ho-Ryun Chung}
\date{}

\begin{document}
\maketitle

\hypertarget{toy-model}{%
\section{Toy model}\label{toy-model}}

\hypertarget{integer-solutions}{%
\subsection{Integer solutions}\label{integer-solutions}}

The toy model with \(L = 3\) binary features has \(\dimstatespace = 8\)
microstates. Using pairwise marginal constraints as information for the
estimation by \ipf we have a coefficient matrix \(\mathbf{C}\) with the
\(D = 12\) marginal constraints in the rows and the
\(\dimstatespace = 8\) microstates in the columns

\begin{Shaded}
\begin{Highlighting}[]
\NormalTok{C }\OtherTok{\textless{}{-}} \FunctionTok{matrix}\NormalTok{(}
  \FunctionTok{c}\NormalTok{(}
    \DecValTok{1}\NormalTok{, }\DecValTok{1}\NormalTok{, }\DecValTok{0}\NormalTok{, }\DecValTok{0}\NormalTok{, }\DecValTok{0}\NormalTok{, }\DecValTok{0}\NormalTok{, }\DecValTok{0}\NormalTok{, }\DecValTok{0}\NormalTok{, }
    \DecValTok{0}\NormalTok{, }\DecValTok{0}\NormalTok{, }\DecValTok{1}\NormalTok{, }\DecValTok{1}\NormalTok{, }\DecValTok{0}\NormalTok{, }\DecValTok{0}\NormalTok{, }\DecValTok{0}\NormalTok{, }\DecValTok{0}\NormalTok{,}
    \DecValTok{0}\NormalTok{, }\DecValTok{0}\NormalTok{, }\DecValTok{0}\NormalTok{, }\DecValTok{0}\NormalTok{, }\DecValTok{1}\NormalTok{, }\DecValTok{1}\NormalTok{, }\DecValTok{0}\NormalTok{, }\DecValTok{0}\NormalTok{,}
    \DecValTok{0}\NormalTok{, }\DecValTok{0}\NormalTok{, }\DecValTok{0}\NormalTok{, }\DecValTok{0}\NormalTok{, }\DecValTok{0}\NormalTok{, }\DecValTok{0}\NormalTok{, }\DecValTok{1}\NormalTok{, }\DecValTok{1}\NormalTok{,}
    \DecValTok{1}\NormalTok{, }\DecValTok{0}\NormalTok{, }\DecValTok{1}\NormalTok{, }\DecValTok{0}\NormalTok{, }\DecValTok{0}\NormalTok{, }\DecValTok{0}\NormalTok{, }\DecValTok{0}\NormalTok{, }\DecValTok{0}\NormalTok{,}
    \DecValTok{0}\NormalTok{, }\DecValTok{1}\NormalTok{, }\DecValTok{0}\NormalTok{, }\DecValTok{1}\NormalTok{, }\DecValTok{0}\NormalTok{, }\DecValTok{0}\NormalTok{, }\DecValTok{0}\NormalTok{, }\DecValTok{0}\NormalTok{,}
    \DecValTok{0}\NormalTok{, }\DecValTok{0}\NormalTok{, }\DecValTok{0}\NormalTok{, }\DecValTok{0}\NormalTok{, }\DecValTok{1}\NormalTok{, }\DecValTok{0}\NormalTok{, }\DecValTok{1}\NormalTok{, }\DecValTok{0}\NormalTok{,}
    \DecValTok{0}\NormalTok{, }\DecValTok{0}\NormalTok{, }\DecValTok{0}\NormalTok{, }\DecValTok{0}\NormalTok{, }\DecValTok{0}\NormalTok{, }\DecValTok{1}\NormalTok{, }\DecValTok{0}\NormalTok{, }\DecValTok{1}\NormalTok{,}
    \DecValTok{1}\NormalTok{, }\DecValTok{0}\NormalTok{, }\DecValTok{0}\NormalTok{, }\DecValTok{0}\NormalTok{, }\DecValTok{1}\NormalTok{, }\DecValTok{0}\NormalTok{, }\DecValTok{0}\NormalTok{, }\DecValTok{0}\NormalTok{,}
    \DecValTok{0}\NormalTok{, }\DecValTok{1}\NormalTok{, }\DecValTok{0}\NormalTok{, }\DecValTok{0}\NormalTok{, }\DecValTok{0}\NormalTok{, }\DecValTok{1}\NormalTok{, }\DecValTok{0}\NormalTok{, }\DecValTok{0}\NormalTok{,}
    \DecValTok{0}\NormalTok{, }\DecValTok{0}\NormalTok{, }\DecValTok{1}\NormalTok{, }\DecValTok{0}\NormalTok{, }\DecValTok{0}\NormalTok{, }\DecValTok{0}\NormalTok{, }\DecValTok{1}\NormalTok{, }\DecValTok{0}\NormalTok{,}
    \DecValTok{0}\NormalTok{, }\DecValTok{0}\NormalTok{, }\DecValTok{0}\NormalTok{, }\DecValTok{1}\NormalTok{, }\DecValTok{0}\NormalTok{, }\DecValTok{0}\NormalTok{, }\DecValTok{0}\NormalTok{, }\DecValTok{1}
\NormalTok{  ),}
  \AttributeTok{ncol =} \DecValTok{8}\NormalTok{,}
  \AttributeTok{byrow =} \ConstantTok{TRUE}
\NormalTok{)}
\end{Highlighting}
\end{Shaded}

The marginal constraints \(\left\{ \widehat m_a\right\}_{a=1,...,D}\)
are given by

\begin{Shaded}
\begin{Highlighting}[]
\NormalTok{mVal }\OtherTok{=} \FunctionTok{c}\NormalTok{(}\DecValTok{30}\NormalTok{, }\DecValTok{14}\NormalTok{, }\DecValTok{27}\NormalTok{, }\DecValTok{29}\NormalTok{, }\DecValTok{20}\NormalTok{, }\DecValTok{24}\NormalTok{, }\DecValTok{48}\NormalTok{, }\DecValTok{8}\NormalTok{, }\DecValTok{39}\NormalTok{, }\DecValTok{18}\NormalTok{, }\DecValTok{29}\NormalTok{, }\DecValTok{14}\NormalTok{)}
\end{Highlighting}
\end{Shaded}

The integer solutions of the so-defined linear system are

\begin{Shaded}
\begin{Highlighting}[]
\NormalTok{sols }\OtherTok{=} \FunctionTok{matrix}\NormalTok{(}
    \FunctionTok{c}\NormalTok{(}
        \DecValTok{15}\NormalTok{, }\DecValTok{15}\NormalTok{, }\DecValTok{5}\NormalTok{, }\DecValTok{9}\NormalTok{, }\DecValTok{24}\NormalTok{, }\DecValTok{3}\NormalTok{, }\DecValTok{24}\NormalTok{, }\DecValTok{5}\NormalTok{,}
        \DecValTok{14}\NormalTok{, }\DecValTok{16}\NormalTok{, }\DecValTok{6}\NormalTok{, }\DecValTok{8}\NormalTok{, }\DecValTok{25}\NormalTok{, }\DecValTok{2}\NormalTok{, }\DecValTok{23}\NormalTok{, }\DecValTok{6}\NormalTok{,}
        \DecValTok{16}\NormalTok{, }\DecValTok{14}\NormalTok{, }\DecValTok{4}\NormalTok{, }\DecValTok{10}\NormalTok{, }\DecValTok{23}\NormalTok{, }\DecValTok{4}\NormalTok{, }\DecValTok{25}\NormalTok{, }\DecValTok{4}\NormalTok{,}
        \DecValTok{13}\NormalTok{, }\DecValTok{17}\NormalTok{, }\DecValTok{7}\NormalTok{, }\DecValTok{7}\NormalTok{, }\DecValTok{26}\NormalTok{, }\DecValTok{1}\NormalTok{, }\DecValTok{22}\NormalTok{, }\DecValTok{7}\NormalTok{,}
        \DecValTok{17}\NormalTok{, }\DecValTok{13}\NormalTok{, }\DecValTok{3}\NormalTok{, }\DecValTok{11}\NormalTok{, }\DecValTok{22}\NormalTok{, }\DecValTok{5}\NormalTok{, }\DecValTok{26}\NormalTok{, }\DecValTok{3}\NormalTok{,}
        \DecValTok{18}\NormalTok{, }\DecValTok{12}\NormalTok{, }\DecValTok{2}\NormalTok{, }\DecValTok{12}\NormalTok{, }\DecValTok{21}\NormalTok{, }\DecValTok{6}\NormalTok{, }\DecValTok{27}\NormalTok{, }\DecValTok{2}\NormalTok{,}
        \DecValTok{12}\NormalTok{, }\DecValTok{18}\NormalTok{, }\DecValTok{8}\NormalTok{, }\DecValTok{6}\NormalTok{, }\DecValTok{27}\NormalTok{, }\DecValTok{0}\NormalTok{, }\DecValTok{21}\NormalTok{, }\DecValTok{8}\NormalTok{,}
        \DecValTok{19}\NormalTok{, }\DecValTok{11}\NormalTok{, }\DecValTok{1}\NormalTok{, }\DecValTok{13}\NormalTok{, }\DecValTok{20}\NormalTok{, }\DecValTok{7}\NormalTok{, }\DecValTok{28}\NormalTok{, }\DecValTok{1}\NormalTok{, }
        \DecValTok{20}\NormalTok{, }\DecValTok{10}\NormalTok{, }\DecValTok{0}\NormalTok{, }\DecValTok{14}\NormalTok{, }\DecValTok{19}\NormalTok{, }\DecValTok{8}\NormalTok{, }\DecValTok{29}\NormalTok{, }\DecValTok{0} 
\NormalTok{    ),}
    \AttributeTok{ncol =} \DecValTok{8}\NormalTok{,}
    \AttributeTok{byrow =} \ConstantTok{TRUE}
\NormalTok{)}
\end{Highlighting}
\end{Shaded}

a matrix with the solutions in the rows and the microstates in the
columns.

Using the solutions we calculate the multinomial probability
\(\text{mult}(N\prob p;\prob u)\)

\begin{Shaded}
\begin{Highlighting}[]
\NormalTok{multinom }\OtherTok{\textless{}{-}} \FunctionTok{apply}\NormalTok{(sols, }\DecValTok{1}\NormalTok{, }\ControlFlowTok{function}\NormalTok{(x) }\FunctionTok{dmultinom}\NormalTok{(x, }\AttributeTok{prob =} \FunctionTok{rep}\NormalTok{(}\DecValTok{1}\NormalTok{, }\DecValTok{8}\NormalTok{), }\AttributeTok{log =} \ConstantTok{TRUE}\NormalTok{))}
\end{Highlighting}
\end{Shaded}

the multinomial coefficients \(W[N\prob p]\)

\begin{Shaded}
\begin{Highlighting}[]
\NormalTok{mult }\OtherTok{\textless{}{-}} \FunctionTok{apply}\NormalTok{(sols, }\DecValTok{1}\NormalTok{, }\ControlFlowTok{function}\NormalTok{(x) }\FunctionTok{lfactorial}\NormalTok{(}\FunctionTok{sum}\NormalTok{(x)) }\SpecialCharTok{{-}} \FunctionTok{sum}\NormalTok{(}\FunctionTok{lfactorial}\NormalTok{(x)))}
\end{Highlighting}
\end{Shaded}

and \(N\) times the entropy \(N H[\prob p]\)

\begin{Shaded}
\begin{Highlighting}[]
\NormalTok{nEntropy }\OtherTok{\textless{}{-}} \FunctionTok{apply}\NormalTok{(}
\NormalTok{    sols, }\DecValTok{1}\NormalTok{,}
    \ControlFlowTok{function}\NormalTok{(x)\{}
        \FunctionTok{sum}\NormalTok{(x) }\SpecialCharTok{*} \FunctionTok{sum}\NormalTok{(}\FunctionTok{ifelse}\NormalTok{(x }\SpecialCharTok{\textgreater{}} \DecValTok{0}\NormalTok{, }\SpecialCharTok{{-}}\NormalTok{x }\SpecialCharTok{/} \FunctionTok{sum}\NormalTok{(x) }\SpecialCharTok{*} \FunctionTok{log}\NormalTok{(x }\SpecialCharTok{/} \FunctionTok{sum}\NormalTok{(x)), }\DecValTok{0}\NormalTok{))}
\NormalTok{    \}}
\NormalTok{)}
\end{Highlighting}
\end{Shaded}

We define the microstates

\begin{Shaded}
\begin{Highlighting}[]
\NormalTok{microstates }\OtherTok{\textless{}{-}} \FunctionTok{t}\NormalTok{(}\FunctionTok{sapply}\NormalTok{(}\DecValTok{0}\SpecialCharTok{:}\DecValTok{7}\NormalTok{,}\ControlFlowTok{function}\NormalTok{(x)\{ }\FunctionTok{as.integer}\NormalTok{(}\FunctionTok{intToBits}\NormalTok{(x))\})[}\DecValTok{1}\SpecialCharTok{:}\DecValTok{3}\NormalTok{, ])}
\NormalTok{microstates }\OtherTok{\textless{}{-}}\NormalTok{ microstates[}\FunctionTok{order}\NormalTok{(microstates[,}\DecValTok{1}\NormalTok{], microstates[,}\DecValTok{2}\NormalTok{], microstates[,}\DecValTok{3}\NormalTok{]), ]}
\end{Highlighting}
\end{Shaded}

and render \textbf{Figure 1}

\begin{Shaded}
\begin{Highlighting}[]
\DocumentationTok{\#\# get color palette}
\FunctionTok{source}\NormalTok{(}\StringTok{"../R/colors.R"}\NormalTok{)}
\FunctionTok{pdf}\NormalTok{(}\StringTok{"../figures/Figure1.pdf"}\NormalTok{, }\AttributeTok{width =} \FloatTok{4.33}\NormalTok{, }\AttributeTok{height =} \FloatTok{1.4}\NormalTok{, }\AttributeTok{pointsize =} \DecValTok{6}\NormalTok{)}
\FunctionTok{par}\NormalTok{(}\AttributeTok{mar =} \FunctionTok{c}\NormalTok{(}\DecValTok{0}\NormalTok{,}\DecValTok{0}\NormalTok{,}\DecValTok{0}\NormalTok{,}\DecValTok{0}\NormalTok{) }\SpecialCharTok{+} \FloatTok{0.1}\NormalTok{)}
\FunctionTok{plot}\NormalTok{(}\ConstantTok{NULL}\NormalTok{, }\AttributeTok{xlim =} \FunctionTok{c}\NormalTok{(}\DecValTok{0}\NormalTok{, }\DecValTok{67}\NormalTok{), }\AttributeTok{ylim =} \FunctionTok{c}\NormalTok{(}\DecValTok{15}\NormalTok{, }\DecValTok{0}\NormalTok{), }\AttributeTok{axes =} \ConstantTok{FALSE}\NormalTok{, }\AttributeTok{ann =} \ConstantTok{FALSE}\NormalTok{, }\AttributeTok{frame =} \ConstantTok{FALSE}\NormalTok{)}

\DocumentationTok{\#\# marginals}
\NormalTok{cr }\OtherTok{\textless{}{-}} \FunctionTok{c}\NormalTok{(cols}\SpecialCharTok{$}\NormalTok{white, cols}\SpecialCharTok{$}\NormalTok{black)}
\NormalTok{off }\OtherTok{=} \DecValTok{0}
\ControlFlowTok{for}\NormalTok{ (i }\ControlFlowTok{in} \DecValTok{1}\SpecialCharTok{:}\DecValTok{12}\NormalTok{)\{}
\NormalTok{  mI }\OtherTok{\textless{}{-}} \FunctionTok{which}\NormalTok{(C[i, ] }\SpecialCharTok{==} \DecValTok{1}\NormalTok{) }
\NormalTok{  fI }\OtherTok{\textless{}{-}} \FunctionTok{which}\NormalTok{(microstates[mI[}\DecValTok{1}\NormalTok{], ] }\SpecialCharTok{==}\NormalTok{ microstates[mI[}\DecValTok{2}\NormalTok{], ])}
\NormalTok{  st }\OtherTok{\textless{}{-}}\NormalTok{ microstates[mI[}\DecValTok{1}\NormalTok{], fI]}
  \FunctionTok{rect}\NormalTok{(}
\NormalTok{    (fI }\SpecialCharTok{{-}} \FloatTok{0.5}\NormalTok{), (i }\SpecialCharTok{{-}} \FloatTok{0.5}\NormalTok{) }\SpecialCharTok{+}\NormalTok{ off, (fI }\SpecialCharTok{+} \FloatTok{0.5}\NormalTok{), (i }\SpecialCharTok{+} \FloatTok{0.5}\NormalTok{) }\SpecialCharTok{+}\NormalTok{ off, }
    \AttributeTok{col =}\NormalTok{ cr[st }\SpecialCharTok{+} \DecValTok{1}\NormalTok{], }\AttributeTok{border =}\NormalTok{ cols}\SpecialCharTok{$}\NormalTok{grey, }\AttributeTok{lwd =} \FloatTok{0.2}
\NormalTok{  )}
  \FunctionTok{text}\NormalTok{(}\FloatTok{0.5}\NormalTok{, i }\SpecialCharTok{+}\NormalTok{ off, mVal[i], }\AttributeTok{pos =} \DecValTok{2}\NormalTok{)}

  \ControlFlowTok{if}\NormalTok{ (i }\SpecialCharTok{\%\%} \DecValTok{4} \SpecialCharTok{==} \DecValTok{0}\NormalTok{)\{}
\NormalTok{      off }\OtherTok{\textless{}{-}}\NormalTok{ off }\SpecialCharTok{+} \FloatTok{0.2}
\NormalTok{  \}}
\NormalTok{\}}

\DocumentationTok{\#\# coefficient matrix C == bipartite graph}
\NormalTok{rr1 }\OtherTok{\textless{}{-}} \FunctionTok{which}\NormalTok{(C }\SpecialCharTok{!=} \DecValTok{0}\NormalTok{, }\AttributeTok{arr.ind =} \ConstantTok{TRUE}\NormalTok{)}
\FunctionTok{segments}\NormalTok{(}\DecValTok{4}\NormalTok{, rr1[,}\DecValTok{1}\NormalTok{] }\SpecialCharTok{+} \FloatTok{0.2} \SpecialCharTok{*}\NormalTok{ ((rr1[, }\DecValTok{1}\NormalTok{] }\SpecialCharTok{{-}} \DecValTok{1}\NormalTok{) }\SpecialCharTok{\%/\%} \DecValTok{4}\NormalTok{), }\DecValTok{9}\NormalTok{, rr1[,}\DecValTok{2}\NormalTok{] }\SpecialCharTok{+} \DecValTok{2}\NormalTok{, }\AttributeTok{col =}\NormalTok{ cols}\SpecialCharTok{$}\NormalTok{red)}
\NormalTok{yy }\OtherTok{\textless{}{-}} \DecValTok{3}\SpecialCharTok{:}\DecValTok{10}
\ControlFlowTok{for}\NormalTok{ (i }\ControlFlowTok{in} \DecValTok{1}\SpecialCharTok{:}\DecValTok{3}\NormalTok{)\{}
  \FunctionTok{rect}\NormalTok{(}
    \DecValTok{9} \SpecialCharTok{+}\NormalTok{ i }\SpecialCharTok{{-}} \FloatTok{0.5}\NormalTok{, (yy }\SpecialCharTok{{-}} \FloatTok{0.5}\NormalTok{), }\DecValTok{9} \SpecialCharTok{+}\NormalTok{ i }\SpecialCharTok{+} \FloatTok{0.5}\NormalTok{, (yy }\SpecialCharTok{+} \FloatTok{0.5}\NormalTok{), }
    \AttributeTok{col =}\NormalTok{ cr[microstates[, i] }\SpecialCharTok{+} \DecValTok{1}\NormalTok{], }\AttributeTok{border =}\NormalTok{ cols}\SpecialCharTok{$}\NormalTok{grey, }\AttributeTok{lwd =} \FloatTok{0.2}
\NormalTok{  )}
\NormalTok{\}}

\DocumentationTok{\#\# integer solution vectors}
\NormalTok{base }\OtherTok{\textless{}{-}} \DecValTok{13}
\FunctionTok{text}\NormalTok{(}\DecValTok{40}\NormalTok{, }\FloatTok{1.4}\NormalTok{, }\StringTok{"Integer solution vectors"}\NormalTok{)}
\FunctionTok{segments}\NormalTok{(}\DecValTok{13}\NormalTok{, }\FloatTok{2.2}\NormalTok{, }\DecValTok{67}\NormalTok{, }\FloatTok{2.2}\NormalTok{)}
\ControlFlowTok{for}\NormalTok{ (i }\ControlFlowTok{in} \DecValTok{1}\SpecialCharTok{:}\FunctionTok{nrow}\NormalTok{(sols))\{}
    \FunctionTok{text}\NormalTok{(base }\SpecialCharTok{+} \DecValTok{6}\NormalTok{, yy, sols[i,], }\AttributeTok{pos =} \DecValTok{2}\NormalTok{)}
\NormalTok{    base }\OtherTok{\textless{}{-}}\NormalTok{ base }\SpecialCharTok{+} \DecValTok{6}
\NormalTok{\}}
\FunctionTok{segments}\NormalTok{(}\DecValTok{13}\NormalTok{, }\FloatTok{10.8}\NormalTok{, }\DecValTok{67}\NormalTok{, }\FloatTok{10.8}\NormalTok{)}

\DocumentationTok{\#\# multinomial probability}
\FunctionTok{text}\NormalTok{(}\FunctionTok{seq}\NormalTok{(}\DecValTok{19}\NormalTok{, }\DecValTok{67}\NormalTok{, }\AttributeTok{by =} \DecValTok{6}\NormalTok{), }\FloatTok{11.6}\NormalTok{, }\FunctionTok{sprintf}\NormalTok{(}\StringTok{"\%.2f"}\NormalTok{, multinom), }\AttributeTok{pos =} \DecValTok{2}\NormalTok{, }\AttributeTok{cex =} \DecValTok{1}\NormalTok{)}
\DocumentationTok{\#\# multinomial coefficient}
\FunctionTok{text}\NormalTok{(}\FunctionTok{seq}\NormalTok{(}\DecValTok{19}\NormalTok{, }\DecValTok{67}\NormalTok{, }\AttributeTok{by =} \DecValTok{6}\NormalTok{), }\FloatTok{12.6}\NormalTok{, }\FunctionTok{sprintf}\NormalTok{(}\StringTok{"\%.2f"}\NormalTok{, mult), }\AttributeTok{pos =} \DecValTok{2}\NormalTok{, }\AttributeTok{cex =} \DecValTok{1}\NormalTok{)}
\DocumentationTok{\#\# N times Entropy}
\FunctionTok{text}\NormalTok{(}\FunctionTok{seq}\NormalTok{(}\DecValTok{19}\NormalTok{, }\DecValTok{67}\NormalTok{, }\AttributeTok{by =} \DecValTok{6}\NormalTok{), }\FloatTok{13.6}\NormalTok{, }\FunctionTok{sprintf}\NormalTok{(}\StringTok{"\%.2f"}\NormalTok{, nEntropy), }\AttributeTok{pos =} \DecValTok{2}\NormalTok{, }\AttributeTok{cex =} \DecValTok{1}\NormalTok{)}
\DocumentationTok{\#\# labels}
\FunctionTok{text}\NormalTok{(}\DecValTok{13}\NormalTok{, }\FloatTok{11.6}\NormalTok{, }\StringTok{"mult p"}\NormalTok{, }\AttributeTok{pos =} \DecValTok{2}\NormalTok{)}
\FunctionTok{text}\NormalTok{(}\DecValTok{13}\NormalTok{, }\FloatTok{12.6}\NormalTok{, }\StringTok{"mult c"}\NormalTok{, }\AttributeTok{pos =} \DecValTok{2}\NormalTok{)}
\FunctionTok{text}\NormalTok{(}\DecValTok{13}\NormalTok{, }\FloatTok{13.6}\NormalTok{, }\StringTok{"NH"}\NormalTok{, }\AttributeTok{pos =} \DecValTok{2}\NormalTok{)}
\FunctionTok{segments}\NormalTok{(}\DecValTok{13}\NormalTok{, }\FloatTok{14.4}\NormalTok{, }\DecValTok{67}\NormalTok{, }\FloatTok{14.4}\NormalTok{)}
\FunctionTok{dev.off}\NormalTok{()}
\end{Highlighting}
\end{Shaded}

\hypertarget{algorithm}{%
\subsection{\texorpdfstring{\ipf algorithm}{algorithm}}\label{algorithm}}

To illustrate the evolution of the estimate \(\maxentP\) during the
iterations in \ipf we use the function \texttt{Ipfp} from the package
\texttt{mipfp}

\begin{Shaded}
\begin{Highlighting}[]
\ControlFlowTok{if}\NormalTok{ (}\SpecialCharTok{!} \FunctionTok{require}\NormalTok{(}\StringTok{"mipfp"}\NormalTok{, }\AttributeTok{quietly =} \ConstantTok{TRUE}\NormalTok{))}
  \FunctionTok{install.packages}\NormalTok{(}\StringTok{"mipfp"}\NormalTok{)}
\FunctionTok{library}\NormalTok{(mipfp)}
\end{Highlighting}
\end{Shaded}

\texttt{Ipfp} takes as arguments

\begin{itemize}
\tightlist
\item
  \texttt{seed} the initial multi-dimensional array to be updated
\item
  \texttt{target.list} a list of dimensions correspond to the marginal
  constraints
\item
  \texttt{target.data} the marginal constraints
\end{itemize}

We defined \texttt{seed} using the uniform distribution (all entries are
\(1/8\)) and \texttt{dim\ =\ c(2,2,2)}

\begin{Shaded}
\begin{Highlighting}[]
\NormalTok{seed }\OtherTok{=} \FunctionTok{array}\NormalTok{(}\DecValTok{1}\SpecialCharTok{/}\DecValTok{8}\NormalTok{, }\AttributeTok{dim =} \FunctionTok{c}\NormalTok{(}\DecValTok{2}\NormalTok{,}\DecValTok{2}\NormalTok{,}\DecValTok{2}\NormalTok{))}
\end{Highlighting}
\end{Shaded}

the \texttt{target.list}

\begin{Shaded}
\begin{Highlighting}[]
\NormalTok{target.list }\OtherTok{\textless{}{-}} \FunctionTok{list}\NormalTok{(}
  \FunctionTok{c}\NormalTok{(}\DecValTok{1}\NormalTok{,}\DecValTok{2}\NormalTok{),}
  \FunctionTok{c}\NormalTok{(}\DecValTok{1}\NormalTok{,}\DecValTok{3}\NormalTok{),}
  \FunctionTok{c}\NormalTok{(}\DecValTok{2}\NormalTok{,}\DecValTok{3}\NormalTok{)}
\NormalTok{)}
\end{Highlighting}
\end{Shaded}

and the \texttt{target.data}

\begin{Shaded}
\begin{Highlighting}[]
\NormalTok{target.data }\OtherTok{\textless{}{-}} \FunctionTok{list}\NormalTok{(}
  \FunctionTok{array}\NormalTok{(}\FunctionTok{c}\NormalTok{(}\DecValTok{30}\NormalTok{,}\DecValTok{27}\NormalTok{,}\DecValTok{14}\NormalTok{,}\DecValTok{29}\NormalTok{) }\SpecialCharTok{/} \DecValTok{100}\NormalTok{, }\AttributeTok{dim =} \FunctionTok{c}\NormalTok{(}\DecValTok{2}\NormalTok{,}\DecValTok{2}\NormalTok{)),}
  \FunctionTok{array}\NormalTok{(}\FunctionTok{c}\NormalTok{(}\DecValTok{20}\NormalTok{,}\DecValTok{48}\NormalTok{,}\DecValTok{24}\NormalTok{,}\DecValTok{8}\NormalTok{) }\SpecialCharTok{/} \DecValTok{100}\NormalTok{, }\AttributeTok{dim =} \FunctionTok{c}\NormalTok{(}\DecValTok{2}\NormalTok{,}\DecValTok{2}\NormalTok{)),}
  \FunctionTok{array}\NormalTok{(}\FunctionTok{c}\NormalTok{(}\DecValTok{39}\NormalTok{, }\DecValTok{29}\NormalTok{, }\DecValTok{18}\NormalTok{,}\DecValTok{14}\NormalTok{) }\SpecialCharTok{/} \DecValTok{100}\NormalTok{, }\AttributeTok{dim =} \FunctionTok{c}\NormalTok{(}\DecValTok{2}\NormalTok{,}\DecValTok{2}\NormalTok{))}
\NormalTok{)}
\end{Highlighting}
\end{Shaded}

We ran \ipf 

\begin{Shaded}
\begin{Highlighting}[]
\NormalTok{ipf }\OtherTok{\textless{}{-}} \FunctionTok{Ipfp}\NormalTok{(seed, target.list, target.data)}
\end{Highlighting}
\end{Shaded}

The results is in \texttt{ipf\$p.hat}.

We recorded the evolution of \(\maxentP\) during the iterations of \ipf

\begin{Shaded}
\begin{Highlighting}[]
\DocumentationTok{\#\# current result}
\NormalTok{result }\OtherTok{\textless{}{-}}\NormalTok{ seed}

\DocumentationTok{\#\# trajectory of p}
\NormalTok{traj }\OtherTok{\textless{}{-}} \FunctionTok{list}\NormalTok{()}
\NormalTok{traj[[}\DecValTok{1}\NormalTok{]] }\OtherTok{=}\NormalTok{ result}
\NormalTok{k }\OtherTok{\textless{}{-}} \DecValTok{2}

\DocumentationTok{\#\# look at the first 4 cycles}
\ControlFlowTok{for}\NormalTok{ (i }\ControlFlowTok{in} \DecValTok{1}\SpecialCharTok{:}\DecValTok{4}\NormalTok{) \{}
\NormalTok{  result.temp }\OtherTok{\textless{}{-}}\NormalTok{ result}
  \ControlFlowTok{for}\NormalTok{ (j }\ControlFlowTok{in} \DecValTok{1}\SpecialCharTok{:}\FunctionTok{length}\NormalTok{(target.list)) \{}
\NormalTok{    temp.sum }\OtherTok{\textless{}{-}} \FunctionTok{apply}\NormalTok{(result, target.list[[j]], sum)}
\NormalTok{    update.factor }\OtherTok{\textless{}{-}} \FunctionTok{ifelse}\NormalTok{(target.data[[j]] }\SpecialCharTok{==} \DecValTok{0} \SpecialCharTok{|}\NormalTok{ temp.sum }\SpecialCharTok{==} 
                              \DecValTok{0}\NormalTok{, }\DecValTok{0}\NormalTok{, target.data[[j]]}\SpecialCharTok{/}\NormalTok{temp.sum)}
\NormalTok{    result }\OtherTok{\textless{}{-}} \FunctionTok{sweep}\NormalTok{(result, target.list[[j]], update.factor, }
                    \AttributeTok{FUN =} \StringTok{"*"}\NormalTok{)}
\NormalTok{    traj[[k]] }\OtherTok{\textless{}{-}}\NormalTok{ result}
\NormalTok{    k }\OtherTok{\textless{}{-}}\NormalTok{ k }\SpecialCharTok{+} \DecValTok{1}
\NormalTok{  \}}
\NormalTok{\}}
\DocumentationTok{\#\# add the final solution of IPF to the trajectory}
\NormalTok{traj[[k]] }\OtherTok{\textless{}{-}}\NormalTok{ ipf}\SpecialCharTok{$}\NormalTok{p.hat}
\end{Highlighting}
\end{Shaded}

We generated a matrix of indices to query the multidimensional arrays

\begin{Shaded}
\begin{Highlighting}[]
\NormalTok{idx }\OtherTok{\textless{}{-}} \FunctionTok{as.matrix}\NormalTok{(}\FunctionTok{expand.grid}\NormalTok{(}\FunctionTok{c}\NormalTok{(}\DecValTok{1}\NormalTok{,}\DecValTok{2}\NormalTok{), }\FunctionTok{c}\NormalTok{(}\DecValTok{1}\NormalTok{,}\DecValTok{2}\NormalTok{), }\FunctionTok{c}\NormalTok{(}\DecValTok{1}\NormalTok{,}\DecValTok{2}\NormalTok{)))}
\end{Highlighting}
\end{Shaded}

Using \texttt{idx} we get the values of \(\maxentP\) during the
different phases

\begin{Shaded}
\begin{Highlighting}[]
\NormalTok{gg }\OtherTok{\textless{}{-}} \FunctionTok{t}\NormalTok{(}\FunctionTok{sapply}\NormalTok{(traj, }\ControlFlowTok{function}\NormalTok{(x) x[idx]))}
\end{Highlighting}
\end{Shaded}

and calculate the \textsc{KL} divergence from the final estimate
\(\maxentP\) obtained by \ipf

\begin{Shaded}
\begin{Highlighting}[]
\NormalTok{dkl }\OtherTok{=} \FunctionTok{sapply}\NormalTok{(traj[}\DecValTok{1}\SpecialCharTok{:}\DecValTok{13}\NormalTok{], }\ControlFlowTok{function}\NormalTok{(x) }\FunctionTok{sum}\NormalTok{(gg[}\DecValTok{14}\NormalTok{, ] }\SpecialCharTok{*} \FunctionTok{log}\NormalTok{(gg[}\DecValTok{14}\NormalTok{,] }\SpecialCharTok{/}\NormalTok{ x)))}
\end{Highlighting}
\end{Shaded}

Finally we render \textbf{Figure 2}

\begin{Shaded}
\begin{Highlighting}[]
\FunctionTok{pdf}\NormalTok{(}\AttributeTok{file =} \StringTok{"../figures/Figure2.pdf"}\NormalTok{, }\AttributeTok{width =} \FloatTok{4.33}\NormalTok{, }\AttributeTok{height =} \FloatTok{2.5}\NormalTok{, }\AttributeTok{pointsize =} \DecValTok{6}\NormalTok{)}

\FunctionTok{layout}\NormalTok{(}
  \FunctionTok{matrix}\NormalTok{(}\FunctionTok{c}\NormalTok{(}\DecValTok{1}\NormalTok{,}\DecValTok{2}\NormalTok{), }\AttributeTok{nrow =} \DecValTok{2}\NormalTok{),}
  \AttributeTok{height =} \FunctionTok{c}\NormalTok{(}\DecValTok{2}\NormalTok{,}\DecValTok{1}\NormalTok{)}
\NormalTok{)}

\FunctionTok{par}\NormalTok{(}\AttributeTok{mar =} \FunctionTok{c}\NormalTok{(}\FloatTok{0.1}\NormalTok{,}\DecValTok{4}\NormalTok{,}\DecValTok{0}\NormalTok{,}\DecValTok{8}\NormalTok{) }\SpecialCharTok{+} \FloatTok{0.1}\NormalTok{)}
\FunctionTok{plot}\NormalTok{(}\ConstantTok{NULL}\NormalTok{, }\AttributeTok{ylim =} \FunctionTok{c}\NormalTok{(}\DecValTok{0}\NormalTok{, }\FloatTok{0.3}\NormalTok{), }\AttributeTok{xlim =} \FunctionTok{c}\NormalTok{(}\DecValTok{1}\NormalTok{,}\DecValTok{13}\NormalTok{), }
     \AttributeTok{ylab =} \StringTok{"IPF estimate"}\NormalTok{, }\AttributeTok{axes =} \ConstantTok{FALSE}\NormalTok{, }\AttributeTok{xlab =} \StringTok{""}
\NormalTok{)}

\FunctionTok{rect}\NormalTok{(}\FloatTok{0.5}\NormalTok{, }\DecValTok{0}\NormalTok{, }\FloatTok{1.5}\NormalTok{, }\FloatTok{0.3}\NormalTok{, }\AttributeTok{col =}\NormalTok{ cols}\SpecialCharTok{$}\NormalTok{white, }\AttributeTok{border =} \ConstantTok{NA}\NormalTok{)}
\FunctionTok{rect}\NormalTok{(}\FloatTok{4.5}\NormalTok{, }\DecValTok{0}\NormalTok{, }\FloatTok{7.5}\NormalTok{, }\FloatTok{0.3}\NormalTok{, }\AttributeTok{col =}\NormalTok{ cols}\SpecialCharTok{$}\NormalTok{white, }\AttributeTok{border =} \ConstantTok{NA}\NormalTok{)}
\FunctionTok{rect}\NormalTok{(}\FloatTok{10.5}\NormalTok{, }\DecValTok{0}\NormalTok{, }\FloatTok{13.5}\NormalTok{, }\FloatTok{0.3}\NormalTok{, }\AttributeTok{col =}\NormalTok{ cols}\SpecialCharTok{$}\NormalTok{white, }\AttributeTok{border =} \ConstantTok{NA}\NormalTok{)}
\FunctionTok{text}\NormalTok{(}\FunctionTok{c}\NormalTok{(}\DecValTok{3}\NormalTok{,}\DecValTok{6}\NormalTok{,}\DecValTok{9}\NormalTok{, }\DecValTok{12}\NormalTok{), }\FunctionTok{rep}\NormalTok{(}\FloatTok{0.28}\NormalTok{, }\DecValTok{4}\NormalTok{), }\FunctionTok{paste0}\NormalTok{(}\DecValTok{1}\SpecialCharTok{:}\DecValTok{4}\NormalTok{, }\StringTok{". cycle"}\NormalTok{))}
\FunctionTok{abline}\NormalTok{(}\AttributeTok{h =}\NormalTok{ gg[}\DecValTok{14}\NormalTok{,], }\AttributeTok{lty =} \DecValTok{3}\NormalTok{, }\AttributeTok{col =} \FunctionTok{unlist}\NormalTok{(cols[}\DecValTok{3}\SpecialCharTok{:}\DecValTok{10}\NormalTok{]))}

\ControlFlowTok{for}\NormalTok{ (i }\ControlFlowTok{in} \DecValTok{1}\SpecialCharTok{:}\DecValTok{8}\NormalTok{)\{}
  \FunctionTok{lines}\NormalTok{(gg[}\DecValTok{1}\SpecialCharTok{:}\DecValTok{13}\NormalTok{,i], }\AttributeTok{col =}\NormalTok{ cols[[i }\SpecialCharTok{+} \DecValTok{2}\NormalTok{]])}
  \FunctionTok{points}\NormalTok{(gg[}\DecValTok{1}\SpecialCharTok{:}\DecValTok{13}\NormalTok{,i], }\AttributeTok{col =}\NormalTok{ cols[[i }\SpecialCharTok{+} \DecValTok{2}\NormalTok{]], }\AttributeTok{pch =} \DecValTok{16}\NormalTok{)}
\NormalTok{\}}
\FunctionTok{axis}\NormalTok{(}\DecValTok{2}\NormalTok{, }\AttributeTok{las =} \DecValTok{1}\NormalTok{)}

\FunctionTok{par}\NormalTok{(}\AttributeTok{xpd =} \ConstantTok{TRUE}\NormalTok{)}
\FunctionTok{text}\NormalTok{(}\FloatTok{13.5}\NormalTok{, gg[}\DecValTok{14}\NormalTok{, ], gg[}\DecValTok{14}\NormalTok{,], }\AttributeTok{col =} \FunctionTok{unlist}\NormalTok{(cols[}\DecValTok{3}\SpecialCharTok{:}\DecValTok{10}\NormalTok{]), }\AttributeTok{pos =} \DecValTok{4}\NormalTok{)}
\FunctionTok{par}\NormalTok{(}\AttributeTok{xpd =} \ConstantTok{FALSE}\NormalTok{)}

\FunctionTok{par}\NormalTok{(}\AttributeTok{mar =} \FunctionTok{c}\NormalTok{(}\DecValTok{2}\NormalTok{,}\DecValTok{4}\NormalTok{,}\DecValTok{0}\NormalTok{,}\DecValTok{8}\NormalTok{) }\SpecialCharTok{+} \FloatTok{0.1}\NormalTok{)}

\FunctionTok{plot}\NormalTok{(dkl, }\AttributeTok{log =} \StringTok{\textquotesingle{}y\textquotesingle{}}\NormalTok{, }\AttributeTok{frame =} \ConstantTok{FALSE}\NormalTok{, }\AttributeTok{xaxt =} \StringTok{"n"}\NormalTok{, }\AttributeTok{ylim =} \FunctionTok{c}\NormalTok{(}\FloatTok{1e{-}7}\NormalTok{,}\DecValTok{1}\NormalTok{), }\AttributeTok{type =} \StringTok{\textquotesingle{}l\textquotesingle{}}\NormalTok{, }\AttributeTok{xlab =} \StringTok{""}\NormalTok{)}
\FunctionTok{points}\NormalTok{(dkl, }\AttributeTok{pch =} \DecValTok{16}\NormalTok{)}
\FunctionTok{axis}\NormalTok{(}\DecValTok{1}\NormalTok{, }\AttributeTok{at =} \DecValTok{1}\SpecialCharTok{:}\DecValTok{13}\NormalTok{, }\AttributeTok{labels =} \DecValTok{0}\SpecialCharTok{:}\DecValTok{12}\NormalTok{)}

\FunctionTok{dev.off}\NormalTok{()}
\end{Highlighting}
\end{Shaded}

\hypertarget{iterative-proportional-fitting-using-the-nhamcs-data}{%
\section{Iterative proportional fitting using the NHAMCS
data}\label{iterative-proportional-fitting-using-the-nhamcs-data}}

We will be using data from the public database of the \emph{National
Center for Health Statistics}. Specifically, we have considered all
entries documented over the years 2003,-,2018 at the emergency
department (\textsc{ed}) by the National Hospital Ambulatory Medical
Care Survey (\textsc{nhamcs})\footnote{Datasets and documentation can be
  downloaded from
  \href{https://www.cdc.gov/nchs/ahcd/datasets_documentation_related.htm\#notices}{https://www.cdc.gov/nchs/}
  for public use.}.

We have concentrated on patients whose fate is known, excluding anyone
who either left against medical advice or left without being seen or
before treatment completion. In addition, anyone who arrived already
dead at the \textsc{ed} was excluded. In this way, \(N=392\,454\)
records remained from which \(44\,293\) were admitted to the hospital. A
patient is considered critical whenever the patient is admitted to an
intensive/critical care unit (\textsc{icu}) or died during the hospital
stay resulting in \(6\,300\) critical cases out of \(44\,293\)
hospitalized patients. Here, we encounter a structural zero, because
non-hospitalized but critical patients are impossible.

In detail, we define \(L = 5\) categorical features:

\begin{itemize}
\tightlist
\item
  \texttt{ageGroup}: age group of a patient with \(q_1 = 6\) states:
  \([0, 15)\:(\alpha_1 = 0)\), \([15,25)\:(\alpha_1 = 1)\),
  \([25, 45)\:(\alpha_1 = 2)\), \([45,65)\:(\alpha_1 = 3)\),
  \([65, 75)\:(\alpha_1 = 4)\), and \([76, 100)\:(\alpha_1 = 5)\)
\item
  \texttt{sex}: sex of patient with \(q_2 = 2\) states: \texttt{female}
  \((\alpha_2 = 0)\) or \texttt{male} \((\alpha_2 = 1)\)
\item
  \texttt{arrivalByAmbulance}: did the patient arrive by ambulance? with
  \(q_3 = 2\) states: \texttt{no} \((\alpha_3 = 0)\) or \texttt{yes}
  \((\alpha_3 = 1)\)
\item
  \texttt{hospitalization}: was the patient hospitalized? with
  \(q_4 = 2\) states: \texttt{no} \((\alpha_4 = 0)\) or \texttt{yes}
  \((\alpha_4 = 1)\)
\item
  \texttt{critical}: was the patient admitted to the intensive care unit
  or died? with \(q_5 = 2\) states: `\texttt{no} \((\alpha_5 = 0)\) or
  \texttt{yes} \((\alpha_5 = 1)\)
\end{itemize}

There are \(\dimstatespace = 6\cdot 2\cdot 2\cdot 2 \cdot 2 = 96\)
microstates. However, from those only \(\vert\mathcal A'\vert=72\) are
not associated with the aforementioned structural zero. We consider
\emph{all} marginal relative frequencies of orders \(\ell = 1,2,3,4\).
In principle we could estimate also the \maxent distribution for
\(\ell = 5\), but due to the reduced state space \(\mathcal A'\), the
equivalence class \([f]_{\ell = 4}\) incorporating all 4-order
constraints is identical to \([f]_{\ell = 5}\) induced by the empirical
distribution \(\prob f\) itself.

\hypertarget{data-preparations}{%
\subsection{Data preparations}\label{data-preparations}}

The data with the empirical counts for the \(\dimstatespace = 96\)
microstates is stored in the file \texttt{../data/empirical.csv}, which
we load into R:

\begin{Shaded}
\begin{Highlighting}[]
\NormalTok{empirical }\OtherTok{\textless{}{-}} \FunctionTok{read.csv}\NormalTok{(}\StringTok{"../data/empirical.csv"}\NormalTok{)}
\end{Highlighting}
\end{Shaded}

Here, we show the first 5 rows of the table:

\begin{verbatim}
##   ageGroup sex arrivalByAmbulance hospitalization critical count
## 1        0   0                  0               0        0 32747
## 2        1   0                  0               0        0 31172
## 3        2   0                  0               0        0 54007
## 4        3   0                  0               0        0 32345
## 5        4   0                  0               0        0  8159
## 6        5   0                  0               0        0  8761
\end{verbatim}

Some of the microstates have not been observed. These are associated
with the structural zero between \texttt{hospitalization} \(=\)
\texttt{no} and \texttt{critical} \(=\) \texttt{yes}:

\begin{verbatim}
##    ageGroup sex arrivalByAmbulance hospitalization critical count
## 49        0   0                  0               0        1     0
## 50        1   0                  0               0        1     0
## 51        2   0                  0               0        1     0
## 52        3   0                  0               0        1     0
## 53        4   0                  0               0        1     0
## 54        5   0                  0               0        1     0
\end{verbatim}

We define the number of features \(L\) and the number of states per
features \(\mathbf q\)

\begin{Shaded}
\begin{Highlighting}[]
\NormalTok{L }\OtherTok{\textless{}{-}} \DecValTok{5}
\NormalTok{q }\OtherTok{\textless{}{-}} \FunctionTok{c}\NormalTok{(}\DecValTok{6}\NormalTok{, }\DecValTok{2}\NormalTok{, }\DecValTok{2}\NormalTok{, }\DecValTok{2}\NormalTok{, }\DecValTok{2}\NormalTok{)}
\end{Highlighting}
\end{Shaded}

From the empirical data we construct a joint distribution \(\prob f\) as
an \texttt{array} with \texttt{dim\ =\ c(6,2,2,2)} \(=\mathbf q\).
First, we extract a matrix \(\mathcal A\) with the
\(\dimstatespace = 96\) microstates in the rows and the \(L = 5\)
features in the columns:

\begin{Shaded}
\begin{Highlighting}[]
\NormalTok{A }\OtherTok{\textless{}{-}} \FunctionTok{as.matrix}\NormalTok{(}
\NormalTok{  empirical[, }
    \FunctionTok{c}\NormalTok{(}
      \StringTok{"ageGroup"}\NormalTok{, }
      \StringTok{"sex"}\NormalTok{, }
      \StringTok{"arrivalByAmbulance"}\NormalTok{, }
      \StringTok{"hospitalization"}\NormalTok{, }
      \StringTok{"critical"}
\NormalTok{    )}
\NormalTok{  ]}
\NormalTok{)}
\end{Highlighting}
\end{Shaded}

As R uses 1-based indexing, we construct from this an index matrix by
adding one to any entry in \(\mathcal A\):

\begin{Shaded}
\begin{Highlighting}[]
\NormalTok{idx }\OtherTok{\textless{}{-}}\NormalTok{ A }\SpecialCharTok{+} \DecValTok{1}
\end{Highlighting}
\end{Shaded}

The empirical count for each microstate is stored in the column
\texttt{count} of the data frame \texttt{empirical}:

\begin{Shaded}
\begin{Highlighting}[]
\NormalTok{h }\OtherTok{\textless{}{-}}\NormalTok{ empirical}\SpecialCharTok{$}\NormalTok{count}
\end{Highlighting}
\end{Shaded}

The total number of records in the data set is given by the sum of the
counts

\begin{Shaded}
\begin{Highlighting}[]
\NormalTok{N }\OtherTok{\textless{}{-}} \FunctionTok{sum}\NormalTok{(h)}
\end{Highlighting}
\end{Shaded}

We define the aforementioned array \(\prob f\) with \texttt{dim} =
\(\mathbf q\)

\begin{Shaded}
\begin{Highlighting}[]
\NormalTok{f }\OtherTok{\textless{}{-}} \FunctionTok{array}\NormalTok{(}
  \DecValTok{1}\NormalTok{,}
  \AttributeTok{dim =}\NormalTok{ q}
\NormalTok{)}
\end{Highlighting}
\end{Shaded}

and fill the array using \texttt{idx} with the counts in \texttt{h}

\begin{Shaded}
\begin{Highlighting}[]
\NormalTok{f[idx] }\OtherTok{=}\NormalTok{ h}
\end{Highlighting}
\end{Shaded}

\hypertarget{estimation-of-distributions-with}{%
\subsection{\texorpdfstring{Estimation of \maxent distributions with
\ipf}{Estimation of distributions with }}\label{estimation-of-distributions-with}}

\hypertarget{computation-of-all-ell-order-marginal-constraints-from-prob-f}{%
\subsubsection{\texorpdfstring{Computation of all \(\ell\)-order
marginal constraints from
\(\prob f\)}{Computation of all \textbackslash ell-order marginal constraints from \textbackslash prob f}}\label{computation-of-all-ell-order-marginal-constraints-from-prob-f}}

Using the empirical frequencies \(\prob f\) for all the
\(\dimstatespace = 96\) we compute the marginal constraints of order
\(\ell\). These correspond to all \(\ell\)-way contingency tables. We
define a function to calculate these constraints

\begin{Shaded}
\begin{Highlighting}[]
\NormalTok{marginalsByOrder }\OtherTok{\textless{}{-}} \ControlFlowTok{function}\NormalTok{(f, ell, L)\{}
  \ControlFlowTok{if}\NormalTok{ (ell }\SpecialCharTok{\textgreater{}} \DecValTok{0} \SpecialCharTok{\&\&}\NormalTok{ ell }\SpecialCharTok{\textless{}}\NormalTok{ L)\{}
\NormalTok{    tuples }\OtherTok{\textless{}{-}} \FunctionTok{combn}\NormalTok{(}\FunctionTok{seq\_len}\NormalTok{(L), ell)}
\NormalTok{    marginals }\OtherTok{\textless{}{-}} \FunctionTok{apply}\NormalTok{(}
\NormalTok{      tuples, }\DecValTok{2}\NormalTok{,}
      \ControlFlowTok{function}\NormalTok{(tuple)\{}
        \FunctionTok{apply}\NormalTok{(f, tuple, sum)}
\NormalTok{      \}}
\NormalTok{    )}
\NormalTok{    list }\OtherTok{\textless{}{-}} \FunctionTok{lapply}\NormalTok{(}\FunctionTok{seq\_len}\NormalTok{(}\FunctionTok{ncol}\NormalTok{(tuples)), }\ControlFlowTok{function}\NormalTok{(i) tuples[, i])}
    
    \FunctionTok{list}\NormalTok{(}\AttributeTok{marginals =}\NormalTok{ marginals, }\AttributeTok{list =}\NormalTok{ list)}
\NormalTok{  \} }\ControlFlowTok{else}\NormalTok{\{}
    \FunctionTok{list}\NormalTok{(}\AttributeTok{marginals =}\NormalTok{ f, }\AttributeTok{list =} \FunctionTok{list}\NormalTok{(}\FunctionTok{seq\_len}\NormalTok{(L)))}
\NormalTok{  \}}
\NormalTok{\}}
\end{Highlighting}
\end{Shaded}

The function takes as argument

\begin{itemize}
\tightlist
\item
  \texttt{f} the empirical frequencies \(\prob f\) stored in a
  multidimensional \texttt{array} with \texttt{dim} \(=\mathbf q\)
\item
  \texttt{ell} the order \(\ell\) of the marginal constraints to compute
\item
  \texttt{L} the number of features \(L\)
\end{itemize}

As we want to compute \emph{all} marginal constraints of order \(\ell\)
we generate all combinations of size \(\ell\) using \(L\) features

\begin{Shaded}
\begin{Highlighting}[]
\NormalTok{tuples }\OtherTok{\textless{}{-}} \FunctionTok{combn}\NormalTok{(}\FunctionTok{seq\_len}\NormalTok{(L), ell)}
\end{Highlighting}
\end{Shaded}

and then compute the marginals

\begin{Shaded}
\begin{Highlighting}[]
\NormalTok{marginals }\OtherTok{\textless{}{-}} \FunctionTok{apply}\NormalTok{(}
\NormalTok{  tuples, }\DecValTok{2}\NormalTok{,}
  \ControlFlowTok{function}\NormalTok{(tuple)\{}
    \FunctionTok{apply}\NormalTok{(f, tuple, sum)}
\NormalTok{  \}}
\NormalTok{)}
\end{Highlighting}
\end{Shaded}

we store a list of feature indices that correspond to the marginal
constraints

\begin{Shaded}
\begin{Highlighting}[]
\NormalTok{list }\OtherTok{\textless{}{-}} \FunctionTok{lapply}\NormalTok{(}\FunctionTok{seq\_len}\NormalTok{(}\FunctionTok{ncol}\NormalTok{(tuples)), }\ControlFlowTok{function}\NormalTok{(i) tuples[, i])}
\end{Highlighting}
\end{Shaded}

Obviously, the marginal constraints are defined for orders
\(\ell = 1, \ldots, L\). However, marginal constraints with order
\(\ell = L\) correspond to the observed distribution \(\prob f\). Thus,
in case \(\ell = L\), we just return \(\prob f\).

\hypertarget{running}{%
\subsubsection{\texorpdfstring{Running \ipf}{Running }}\label{running}}

We defined \texttt{seed} using the uniform distribution (all entries are
1) and \texttt{dim} \(=\mathbf q\)

\begin{Shaded}
\begin{Highlighting}[]
\NormalTok{seed }\OtherTok{\textless{}{-}} \FunctionTok{array}\NormalTok{(}
  \DecValTok{1}\NormalTok{,}
  \AttributeTok{dim =}\NormalTok{ q}
\NormalTok{)}
\end{Highlighting}
\end{Shaded}

\textbf{order \(\ell = 1\) marginal constraints}

\begin{Shaded}
\begin{Highlighting}[]
\NormalTok{m1 }\OtherTok{\textless{}{-}} \FunctionTok{marginalsByOrder}\NormalTok{(}
  \AttributeTok{f =}\NormalTok{ f, }
  \AttributeTok{ell =} \DecValTok{1}\NormalTok{, }
  \AttributeTok{L =}\NormalTok{ L}
\NormalTok{)}
\NormalTok{phat1 }\OtherTok{\textless{}{-}} \FunctionTok{Ipfp}\NormalTok{(}
  \AttributeTok{seed =}\NormalTok{ seed,}
  \AttributeTok{target.list =}\NormalTok{ m1}\SpecialCharTok{$}\NormalTok{list,}
  \AttributeTok{target.data =}\NormalTok{ m1}\SpecialCharTok{$}\NormalTok{marginals,}
  \DocumentationTok{\#\# verbose}
  \AttributeTok{print =} \ConstantTok{TRUE}              
\NormalTok{)}
\end{Highlighting}
\end{Shaded}

\begin{verbatim}
## Margins consistency checked!
## ... ITER 1 
##        stoping criterion: 45162.5 
## ... ITER 2 
##        stoping criterion: 7.275958e-12 
## Convergence reached after 2 iterations!
\end{verbatim}

\ipf stops after the first iteration. Trivially, first order constraints
assume independence, i.e.~the joint distribution can be calculated in
one step. \texttt{Ipfp} returns

\begin{verbatim}
## List of 7
##  $ x.hat        : num [1:6, 1:2, 1:2, 1:2, 1:2] 30734 24333 45164 33460 9975 ...
##  $ p.hat        : num [1:6, 1:2, 1:2, 1:2, 1:2] 0.0783 0.062 0.1151 0.0853 0.0254 ...
##  $ conv         : logi TRUE
##  $ error.margins: num [1:5] 3.64e-12 2.91e-11 0.00 0.00 0.00
##  $ evol.stp.crit: num [1:2] 4.52e+04 7.28e-12
##  $ method       : chr "ipfp"
##  $ call         : language Ipfp(seed = seed, target.list = m1$list, target.data = m1$marginals, print = TRUE)
##  - attr(*, "class")= chr [1:2] "list" "mipfp"
\end{verbatim}

a list of 7 elements, where \texttt{\$\ p.hat} contains the estimated
joint probability distribution \(\maxentP^{(\ell = 1)}\).

\textbf{order \(\ell = 2\) marginal constraints}

\begin{Shaded}
\begin{Highlighting}[]
\NormalTok{m2 }\OtherTok{\textless{}{-}} \FunctionTok{marginalsByOrder}\NormalTok{(}
  \AttributeTok{f =}\NormalTok{ f, }
  \AttributeTok{ell =} \DecValTok{2}\NormalTok{, }
  \AttributeTok{L =}\NormalTok{ L}
\NormalTok{)}
\NormalTok{phat2 }\OtherTok{\textless{}{-}} \FunctionTok{Ipfp}\NormalTok{(}
  \AttributeTok{seed =}\NormalTok{ seed,}
  \AttributeTok{target.list =}\NormalTok{ m2}\SpecialCharTok{$}\NormalTok{list,}
  \AttributeTok{target.data =}\NormalTok{ m2}\SpecialCharTok{$}\NormalTok{marginals}
\NormalTok{)}
\end{Highlighting}
\end{Shaded}

\textbf{order \(\ell = 3\) marginal constraints}

\begin{Shaded}
\begin{Highlighting}[]
\NormalTok{m3 }\OtherTok{\textless{}{-}} \FunctionTok{marginalsByOrder}\NormalTok{(}
  \AttributeTok{f =}\NormalTok{ f, }
  \AttributeTok{ell =} \DecValTok{3}\NormalTok{, }
  \AttributeTok{L =}\NormalTok{ L}
\NormalTok{)}
\NormalTok{phat3 }\OtherTok{\textless{}{-}} \FunctionTok{Ipfp}\NormalTok{(}
  \AttributeTok{seed =}\NormalTok{ seed,}
  \AttributeTok{target.list =}\NormalTok{ m3}\SpecialCharTok{$}\NormalTok{list,}
  \AttributeTok{target.data =}\NormalTok{ m3}\SpecialCharTok{$}\NormalTok{marginals}
\NormalTok{)}
\end{Highlighting}
\end{Shaded}

\textbf{order \(\ell = 4\) marginal constraints}

\begin{Shaded}
\begin{Highlighting}[]
\NormalTok{m4 }\OtherTok{\textless{}{-}} \FunctionTok{marginalsByOrder}\NormalTok{(}
  \AttributeTok{f =}\NormalTok{ f, }
  \AttributeTok{ell =} \DecValTok{4}\NormalTok{, }
  \AttributeTok{L =}\NormalTok{ L}
\NormalTok{)}
\NormalTok{phat4 }\OtherTok{=} \FunctionTok{Ipfp}\NormalTok{(}
  \AttributeTok{seed =}\NormalTok{ seed,}
  \AttributeTok{target.list =}\NormalTok{ m4}\SpecialCharTok{$}\NormalTok{list,}
  \AttributeTok{target.data =}\NormalTok{ m4}\SpecialCharTok{$}\NormalTok{marginals}
\NormalTok{)}
\end{Highlighting}
\end{Shaded}

As expect marginal constraints of order \(\ell = 4\) give back the
empirical distribution \(\prob f\)

\begin{Shaded}
\begin{Highlighting}[]
\FunctionTok{all.equal}\NormalTok{(f, phat4}\SpecialCharTok{$}\NormalTok{x.hat)}
\end{Highlighting}
\end{Shaded}

\begin{verbatim}
## [1] TRUE
\end{verbatim}

We store the four \maxent distributions in a \texttt{data.frame} and
save it to file \texttt{../data/TableS2.tsv}

\begin{Shaded}
\begin{Highlighting}[]
\NormalTok{maxentFile }\OtherTok{\textless{}{-}} \StringTok{"../data/TableS2.tsv"}
\NormalTok{maxentP }\OtherTok{\textless{}{-}} \FunctionTok{data.frame}\NormalTok{(}
\NormalTok{  empirical[, }\DecValTok{1}\SpecialCharTok{:}\DecValTok{5}\NormalTok{],}
  \AttributeTok{one =}\NormalTok{ phat1}\SpecialCharTok{$}\NormalTok{p.hat,}
  \AttributeTok{two =}\NormalTok{ phat2}\SpecialCharTok{$}\NormalTok{p.hat,}
  \AttributeTok{three =}\NormalTok{ phat3}\SpecialCharTok{$}\NormalTok{p.hat,}
  \AttributeTok{four =}\NormalTok{ phat4}\SpecialCharTok{$}\NormalTok{p.hat,}
  \AttributeTok{empirical =}\NormalTok{ empirical}\SpecialCharTok{$}\NormalTok{count }\SpecialCharTok{/} \FunctionTok{sum}\NormalTok{(empirical}\SpecialCharTok{$}\NormalTok{count)}
\NormalTok{)}
\FunctionTok{write.table}\NormalTok{(maxentP, }\AttributeTok{file =}\NormalTok{ maxentFile, }\AttributeTok{sep =} \StringTok{\textquotesingle{}}\SpecialCharTok{\textbackslash{}t}\StringTok{\textquotesingle{}}\NormalTok{, }\AttributeTok{quote =} \ConstantTok{TRUE}\NormalTok{, }\AttributeTok{row.names =} \ConstantTok{FALSE}\NormalTok{)}
\end{Highlighting}
\end{Shaded}

\hypertarget{subsampling}{%
\subsection{Subsampling}\label{subsampling}}

We assume that the empirical distribution \(\prob f\) describes a
``population'' of infinite size, i.e.~it corresponds to a fictitious
asymptotic distribution. From this ``asymptotic'' distribution we sample
data sets of size \(N_s\) with replacement. These data sets are
conveniently summarized by the frequencies of the 72 possible
microstates (\texttt{fs}). To prevent infinities during the calculation
of the \textsc{kl} divergence we add a pseudocount of unity to all 72
admissible microstates.

\hypertarget{ell-uniform-order-marginal-constraints}{%
\subsubsection{\texorpdfstring{\(\ell\)-uniform order marginal
constraints}{\textbackslash ell-uniform order marginal constraints}}\label{ell-uniform-order-marginal-constraints}}

We estimate model probability distributions \(\prob p^{(s,\ell)}\) for
sets of marginal constraints encompassing \emph{all} marginal
constraints of a given order \(\ell = 1,2,3,4\). Model distributions
\(\maxentP_{s}^{(\ell)}\) obtained with \(\ell=4,5\) coincide due to the
structural zero between hospitalization and criticality.

We will subsample (with replacement) from the empirical distribution
\(\prob f\). For this we define a function

\begin{Shaded}
\begin{Highlighting}[]
\NormalTok{subSampleUniform }\OtherTok{\textless{}{-}} \ControlFlowTok{function}\NormalTok{(subSampleSize, }\AttributeTok{ells =} \FunctionTok{c}\NormalTok{(}\DecValTok{1}\NormalTok{,}\DecValTok{2}\NormalTok{,}\DecValTok{3}\NormalTok{))\{}
  
  \DocumentationTok{\#\# define the multi{-}dimensional "empirical" distribution}
\NormalTok{  fs }\OtherTok{\textless{}{-}} \FunctionTok{array}\NormalTok{(}
    \DecValTok{1}\NormalTok{,}
    \AttributeTok{dim =}\NormalTok{ q}
\NormalTok{  )}
  \DocumentationTok{\#\# fill it with multinomial samples according to the empirical}
  \DocumentationTok{\#\# multinomial probabilitites stored in f}
\NormalTok{  fs[idx] }\OtherTok{\textless{}{-}} \FunctionTok{rmultinom}\NormalTok{(}
    \AttributeTok{n =} \DecValTok{1}\NormalTok{, }
    \AttributeTok{size =}\NormalTok{ subSampleSize, }
    \AttributeTok{prob =}\NormalTok{ h}
\NormalTok{  )}
  
  \DocumentationTok{\#\# regularize all entries, where the original f was non{-}zero}
\NormalTok{  fs[f }\SpecialCharTok{\textgreater{}} \DecValTok{0}\NormalTok{] }\OtherTok{=}\NormalTok{ fs[f }\SpecialCharTok{\textgreater{}} \DecValTok{0}\NormalTok{] }\SpecialCharTok{+} \DecValTok{1}
  
  \DocumentationTok{\#\# compute the IPF models for the orders stored in ells}
\NormalTok{  res }\OtherTok{\textless{}{-}} \FunctionTok{sapply}\NormalTok{(}
\NormalTok{    ells,}
    \ControlFlowTok{function}\NormalTok{(ell)\{}
      \DocumentationTok{\#\# order ell marginal constraints}
\NormalTok{      m }\OtherTok{\textless{}{-}} \FunctionTok{marginalsByOrder}\NormalTok{(fs, ell, L)}
      
      \DocumentationTok{\#\# define the seed}
\NormalTok{      seed }\OtherTok{\textless{}{-}} \FunctionTok{array}\NormalTok{(}
        \DecValTok{1}\NormalTok{,}
        \AttributeTok{dim =}\NormalTok{ q}
\NormalTok{      )}
      
      \DocumentationTok{\#\# run IPF}
\NormalTok{      p.hat }\OtherTok{=} \FunctionTok{Ipfp}\NormalTok{(}
        \AttributeTok{seed =}\NormalTok{ seed, }
        \AttributeTok{target.list =}\NormalTok{ m}\SpecialCharTok{$}\NormalTok{list,}
        \AttributeTok{target.data =}\NormalTok{ m}\SpecialCharTok{$}\NormalTok{marginals,}
        \AttributeTok{iter =} \DecValTok{10000}
\NormalTok{      )}\SpecialCharTok{$}\NormalTok{p.hat}
      
      \DocumentationTok{\#\# calculate KL divergence}
      \FunctionTok{sum}\NormalTok{(}\FunctionTok{ifelse}\NormalTok{(f }\SpecialCharTok{\textgreater{}} \DecValTok{0}\NormalTok{, f }\SpecialCharTok{*}\NormalTok{ (}\FunctionTok{log}\NormalTok{(f) }\SpecialCharTok{{-}} \FunctionTok{log}\NormalTok{(p.hat)), }\DecValTok{0}\NormalTok{))}
\NormalTok{    \}}
\NormalTok{  )}
  \DocumentationTok{\#\# "empirical"}
  \FunctionTok{c}\NormalTok{(}
\NormalTok{    res,}
    \FunctionTok{sum}\NormalTok{(}\FunctionTok{ifelse}\NormalTok{(f }\SpecialCharTok{\textgreater{}} \DecValTok{0}\NormalTok{, f }\SpecialCharTok{*}\NormalTok{ (}\FunctionTok{log}\NormalTok{(f) }\SpecialCharTok{{-}} \FunctionTok{log}\NormalTok{(fs }\SpecialCharTok{/} \FunctionTok{sum}\NormalTok{(fs))), }\DecValTok{0}\NormalTok{))}
\NormalTok{  )}
    
\NormalTok{\}}
\end{Highlighting}
\end{Shaded}

Every ``model'' uses the complete set of possibly redundant marginal
constraints.

To get reproducible results we set a \texttt{seed}

\begin{Shaded}
\begin{Highlighting}[]
\NormalTok{seed }\OtherTok{=} \DecValTok{20220308}
\FunctionTok{set.seed}\NormalTok{(seed)}
\end{Highlighting}
\end{Shaded}

and sample for each of the subsample sizes 20000 subsamples

\begin{Shaded}
\begin{Highlighting}[]
\NormalTok{Ns }\OtherTok{\textless{}{-}} \FunctionTok{c}\NormalTok{(}\DecValTok{5000}\NormalTok{, }\FunctionTok{seq}\NormalTok{(}\DecValTok{10000}\NormalTok{, }\DecValTok{390000}\NormalTok{, }\AttributeTok{by =} \DecValTok{10000}\NormalTok{), N)}
\NormalTok{nSamples }\OtherTok{\textless{}{-}} \DecValTok{20000}
\end{Highlighting}
\end{Shaded}

For each of these subsamples we compute the KL divergence from the
empirical \(\prob f\) for the four different orders \(\ell = 1,2,3,4\)
of marginal constrains. We perform the subsampling in parallel using the
function \texttt{mclapply} of the package \texttt{parallel}

\begin{Shaded}
\begin{Highlighting}[]
\ControlFlowTok{if}\NormalTok{ (}\SpecialCharTok{!} \FunctionTok{require}\NormalTok{(}\StringTok{"parallel"}\NormalTok{, }\AttributeTok{quietly =} \ConstantTok{TRUE}\NormalTok{))}
  \FunctionTok{install.packages}\NormalTok{(}\StringTok{"parallel"}\NormalTok{)}
\FunctionTok{library}\NormalTok{(parallel)}
\end{Highlighting}
\end{Shaded}

using \texttt{mc.cores} = 256 cores:

\begin{Shaded}
\begin{Highlighting}[]
\NormalTok{dklUniform }\OtherTok{\textless{}{-}} \FunctionTok{lapply}\NormalTok{(}
\NormalTok{  Ns,}
  \ControlFlowTok{function}\NormalTok{(subSampleSize)\{}
    
\NormalTok{    res }\OtherTok{\textless{}{-}} \FunctionTok{mclapply}\NormalTok{(}
      \DecValTok{1}\SpecialCharTok{:}\NormalTok{nSamples,}
      \ControlFlowTok{function}\NormalTok{(i)\{}
        \FunctionTok{subSampleUniform}\NormalTok{(subSampleSize)}
\NormalTok{      \},}
      \AttributeTok{mc.cores =} \DecValTok{256}
\NormalTok{    )}
    
    \FunctionTok{matrix}\NormalTok{(}\FunctionTok{unlist}\NormalTok{(res), }\AttributeTok{ncol =} \DecValTok{4}\NormalTok{, }\AttributeTok{byrow =} \ConstantTok{TRUE}\NormalTok{)}
\NormalTok{  \}}
\NormalTok{)}
\end{Highlighting}
\end{Shaded}

The results are saved in a file \texttt{../data/dklUniform.RData}

\begin{Shaded}
\begin{Highlighting}[]
\NormalTok{dklFile }\OtherTok{\textless{}{-}} \StringTok{"../data/dklUniform.RData"}
\FunctionTok{save}\NormalTok{(seed, dklUniform, }\AttributeTok{file =}\NormalTok{ dklFile)}
\end{Highlighting}
\end{Shaded}

For convenience, we provide a R-script in
\texttt{../R/uniformOrderConstraints.R} that performs the subsampling,
estimation of the models, and calculation of the \textsc{kl} divergence.
The results can be loaded by sourcing the corresponding R-script.

\hypertarget{figure-3}{%
\paragraph{Figure 3}\label{figure-3}}

This script produces Figure 3. We generate/load the subsample data
stored in list \texttt{dklUniform} by sourcing the file
\texttt{../R/subSample.R}

\begin{Shaded}
\begin{Highlighting}[]
\FunctionTok{source}\NormalTok{(}\StringTok{"../R/uniformOrderConstraints.R"}\NormalTok{)}
\end{Highlighting}
\end{Shaded}

In addition we source our color palette

\begin{Shaded}
\begin{Highlighting}[]
\FunctionTok{source}\NormalTok{(}\StringTok{"../R/colors.R"}\NormalTok{)}
\end{Highlighting}
\end{Shaded}

Compute the average KL-divergences per subsample size and per
\(\maxentP_{s}^{(\ell)}\) model

\begin{Shaded}
\begin{Highlighting}[]
\NormalTok{KLavg }\OtherTok{\textless{}{-}} \FunctionTok{sapply}\NormalTok{(}
\NormalTok{  dklUniform, }
\NormalTok{  colMeans  }
\NormalTok{)}
\end{Highlighting}
\end{Shaded}

\(\prob p^{(s,\ell)}\)

Render the Figure

\begin{Shaded}
\begin{Highlighting}[]
\FunctionTok{pdf}\NormalTok{(}\StringTok{"../figures/Figure3.pdf"}\NormalTok{, }\AttributeTok{width =} \FloatTok{7.08}\NormalTok{, }\AttributeTok{height =} \FloatTok{2.5}\NormalTok{, }\AttributeTok{pointsize =} \DecValTok{8}\NormalTok{)}
\FunctionTok{par}\NormalTok{(}\AttributeTok{mar =} \FunctionTok{c}\NormalTok{(}\DecValTok{3}\NormalTok{,}\DecValTok{3}\NormalTok{,}\DecValTok{1}\NormalTok{,}\DecValTok{0}\NormalTok{) }\SpecialCharTok{+} \FloatTok{0.1}\NormalTok{)}
\FunctionTok{plot}\NormalTok{(}
  \DecValTok{1}\SpecialCharTok{:}\FunctionTok{ncol}\NormalTok{(KLavg), KLavg[}\DecValTok{2}\NormalTok{, ], }
  \AttributeTok{ylim =} \FunctionTok{c}\NormalTok{(}\FloatTok{1e{-}4}\NormalTok{, }\FloatTok{1e{-}2}\NormalTok{), }\AttributeTok{log =} \StringTok{"y"}\NormalTok{, }
  \AttributeTok{col =}\NormalTok{ cols}\SpecialCharTok{$}\NormalTok{yellow, }\AttributeTok{frame =} \ConstantTok{FALSE}\NormalTok{, }
  \AttributeTok{xlim =} \FunctionTok{c}\NormalTok{(}\DecValTok{0}\NormalTok{, }\FunctionTok{ncol}\NormalTok{(KLavg) }\SpecialCharTok{+} \DecValTok{1}\NormalTok{), }
  \AttributeTok{pch =} \DecValTok{16}\NormalTok{, }\AttributeTok{cex =} \FloatTok{0.5}\NormalTok{, }
  \AttributeTok{axes =} \ConstantTok{FALSE}\NormalTok{, }\AttributeTok{ann =} \ConstantTok{FALSE}\NormalTok{, }\AttributeTok{xaxs =} \StringTok{"i"}
\NormalTok{) }
\FunctionTok{lines}\NormalTok{(}\DecValTok{1}\SpecialCharTok{:}\FunctionTok{ncol}\NormalTok{(KLavg), KLavg[}\DecValTok{2}\NormalTok{,], }\AttributeTok{col =}\NormalTok{ cols}\SpecialCharTok{$}\NormalTok{yellow)}
\FunctionTok{points}\NormalTok{(}\DecValTok{1}\SpecialCharTok{:}\FunctionTok{ncol}\NormalTok{(KLavg), KLavg[}\DecValTok{3}\NormalTok{, ], }\AttributeTok{col =}\NormalTok{ cols}\SpecialCharTok{$}\NormalTok{red, }\AttributeTok{pch =} \DecValTok{16}\NormalTok{, }\AttributeTok{cex =} \FloatTok{0.5}\NormalTok{) }\DocumentationTok{\#\# IPF{-}3}
\FunctionTok{lines}\NormalTok{(}\DecValTok{1}\SpecialCharTok{:}\FunctionTok{ncol}\NormalTok{(KLavg), KLavg[}\DecValTok{3}\NormalTok{, ], }\AttributeTok{col =}\NormalTok{ cols}\SpecialCharTok{$}\NormalTok{red)}
\FunctionTok{points}\NormalTok{(}\DecValTok{1}\SpecialCharTok{:}\FunctionTok{ncol}\NormalTok{(KLavg), KLavg[}\DecValTok{4}\NormalTok{,], }\AttributeTok{col =}\NormalTok{ cols}\SpecialCharTok{$}\NormalTok{purple, }\AttributeTok{pch =} \DecValTok{16}\NormalTok{, }\AttributeTok{cex =} \FloatTok{0.5}\NormalTok{) }\DocumentationTok{\#\# IPF{-}4}
\FunctionTok{lines}\NormalTok{(}\DecValTok{1}\SpecialCharTok{:}\FunctionTok{ncol}\NormalTok{(KLavg), KLavg[}\DecValTok{4}\NormalTok{,], }\AttributeTok{col =}\NormalTok{ cols}\SpecialCharTok{$}\NormalTok{purple)}

\FunctionTok{axis}\NormalTok{(}\DecValTok{2}\NormalTok{, }\AttributeTok{at =} \FunctionTok{c}\NormalTok{(}\FloatTok{1e{-}7}\NormalTok{, }\FloatTok{1e{-}6}\NormalTok{, }\FloatTok{1e{-}5}\NormalTok{, }\FloatTok{1e{-}4}\NormalTok{, }\FloatTok{1e{-}3}\NormalTok{, }\FloatTok{1e{-}2}\NormalTok{), }\AttributeTok{las =} \DecValTok{1}\NormalTok{)}
\FunctionTok{axis}\NormalTok{(}\DecValTok{1}\NormalTok{, }\AttributeTok{at =} \DecValTok{1}\SpecialCharTok{:}\FunctionTok{ncol}\NormalTok{(KLavg), }\AttributeTok{labels =} \FunctionTok{as.integer}\NormalTok{(Ns) }\SpecialCharTok{/} \DecValTok{1000}\NormalTok{)}
\FunctionTok{par}\NormalTok{(}\AttributeTok{xpd =} \ConstantTok{TRUE}\NormalTok{)}
\FunctionTok{legend}\NormalTok{(}
  \FunctionTok{ncol}\NormalTok{(KLavg) }\SpecialCharTok{{-}} \DecValTok{1}\NormalTok{, }
  \FunctionTok{max}\NormalTok{(KLavg[}\SpecialCharTok{{-}}\DecValTok{1}\NormalTok{,]), }
  \FunctionTok{c}\NormalTok{(}\StringTok{"$}\SpecialCharTok{\textbackslash{}\textbackslash{}}\StringTok{maxentP\^{}\{(2)\}$"}\NormalTok{, }\StringTok{"$}\SpecialCharTok{\textbackslash{}\textbackslash{}}\StringTok{maxentP\^{}\{(3)\}$"}\NormalTok{, }\StringTok{"$}\SpecialCharTok{\textbackslash{}\textbackslash{}}\StringTok{maxentP\^{}\{(4)\}$"}\NormalTok{), }
  \AttributeTok{col =} \FunctionTok{c}\NormalTok{(cols}\SpecialCharTok{$}\NormalTok{yellow, cols}\SpecialCharTok{$}\NormalTok{red, cols}\SpecialCharTok{$}\NormalTok{purple), }
  \AttributeTok{lty =} \DecValTok{1}\NormalTok{, }\AttributeTok{pch =} \DecValTok{16}\NormalTok{, }\AttributeTok{bty =} \StringTok{"n"}\NormalTok{, }\AttributeTok{cex =} \FloatTok{0.5}
\NormalTok{)}
\FunctionTok{par}\NormalTok{(}\AttributeTok{xpd =} \ConstantTok{FALSE}\NormalTok{)}
\FunctionTok{dev.off}\NormalTok{()}
\end{Highlighting}
\end{Shaded}

\hypertarget{mixed-order-marginal-constraints}{%
\subsubsection{Mixed-order marginal
constraints}\label{mixed-order-marginal-constraints}}

We estimate model probability distributions \(\maxentP^{s, k}\) for all
possible summary statistics, where each of the \(L=5\) features belongs
to at least one cluster of features. We generate all combinations of
marginal constraints of mixed order encompassing at least one constraint
for each of the \(L = 5\) features using the function
\texttt{getCombinations} provided in \texttt{../R/getCombinations.R}.
This function takes two arguments:

\begin{itemize}
\tightlist
\item
  \texttt{L} the number of features
\item
  \texttt{upTo} the maximal order of constraints to be considered
\end{itemize}

We evaluate \texttt{getCombinations} this by setting \texttt{L} = 5 and
\texttt{upTo} = 4

\begin{Shaded}
\begin{Highlighting}[]
\FunctionTok{source}\NormalTok{(}\StringTok{"../R/getCombinations.R"}\NormalTok{)}
\NormalTok{combinations }\OtherTok{\textless{}{-}} \FunctionTok{getCombinations}\NormalTok{(}\AttributeTok{L =}\NormalTok{ L, }\AttributeTok{upTo =} \DecValTok{4}\NormalTok{)}
\NormalTok{nn }\OtherTok{\textless{}{-}}\NormalTok{ combinations}\SpecialCharTok{$}\NormalTok{nn}
\NormalTok{combinations }\OtherTok{\textless{}{-}}\NormalTok{ combinations}\SpecialCharTok{$}\NormalTok{combinations}
\end{Highlighting}
\end{Shaded}

Each combination is a \texttt{list} with four sublists, each enumerating
the indices of the marginal constraints to be considered.

Again, we will subsample (with replacement) from the empirical
distribution \(\prob f\). For this we define a function

\begin{Shaded}
\begin{Highlighting}[]
\NormalTok{subSampleMixed }\OtherTok{\textless{}{-}} \ControlFlowTok{function}\NormalTok{(subSampleSize)\{}
  \DocumentationTok{\#\# define the multi{-}dimensional "empirical" distribution}
\NormalTok{  fs }\OtherTok{\textless{}{-}} \FunctionTok{array}\NormalTok{(}
    \DecValTok{1}\NormalTok{,}
    \AttributeTok{dim =}\NormalTok{ q}
\NormalTok{  )}
  
  \DocumentationTok{\#\# fill it with multinomial samples according to the empirical}
  \DocumentationTok{\#\# multinomial probabilities stored in f}
\NormalTok{  fs[idx] }\OtherTok{\textless{}{-}} \FunctionTok{rmultinom}\NormalTok{(}
    \AttributeTok{n =} \DecValTok{1}\NormalTok{, }
    \AttributeTok{size =}\NormalTok{ subSampleSize, }
    \AttributeTok{prob =}\NormalTok{ h}
\NormalTok{  )}
  
  \DocumentationTok{\#\# regularize all entries, where the original f was non{-}zero}
\NormalTok{  fs[f }\SpecialCharTok{\textgreater{}} \DecValTok{0}\NormalTok{] }\OtherTok{\textless{}{-}}\NormalTok{ fs[f }\SpecialCharTok{\textgreater{}} \DecValTok{0}\NormalTok{] }\SpecialCharTok{+} \DecValTok{1}
\NormalTok{  margByOrder }\OtherTok{\textless{}{-}} \FunctionTok{lapply}\NormalTok{(}
    \FunctionTok{c}\NormalTok{(}\DecValTok{1}\NormalTok{,}\DecValTok{2}\NormalTok{,}\DecValTok{3}\NormalTok{,}\DecValTok{4}\NormalTok{),}
    \ControlFlowTok{function}\NormalTok{(ell)\{}
      \FunctionTok{marginalsByOrder}\NormalTok{(}
        \AttributeTok{f =}\NormalTok{ fs,}
        \AttributeTok{ell =}\NormalTok{ ell,}
        \AttributeTok{L =}\NormalTok{ L}
\NormalTok{      )}
\NormalTok{    \}}
\NormalTok{  )}
  
  \DocumentationTok{\#\# estimate the MaxEnt distribution}
  \DocumentationTok{\#\# for each set of marginal constraints}
\NormalTok{  PP }\OtherTok{\textless{}{-}} \FunctionTok{mclapply}\NormalTok{(}
\NormalTok{    combinations,}
    \ControlFlowTok{function}\NormalTok{(comb)\{}
\NormalTok{      seed }\OtherTok{\textless{}{-}} \FunctionTok{array}\NormalTok{(}
        \DecValTok{1}\NormalTok{,}
        \AttributeTok{dim =}\NormalTok{ q}
\NormalTok{      )}
      \DocumentationTok{\#\# run IPF}
\NormalTok{      target.list }\OtherTok{\textless{}{-}} \FunctionTok{c}\NormalTok{(}
\NormalTok{        margByOrder[[}\DecValTok{1}\NormalTok{]]}\SpecialCharTok{$}\NormalTok{list[comb[[}\DecValTok{1}\NormalTok{]]],}
\NormalTok{        margByOrder[[}\DecValTok{2}\NormalTok{]]}\SpecialCharTok{$}\NormalTok{list[comb[[}\DecValTok{2}\NormalTok{]]],}
\NormalTok{        margByOrder[[}\DecValTok{3}\NormalTok{]]}\SpecialCharTok{$}\NormalTok{list[comb[[}\DecValTok{3}\NormalTok{]]],}
\NormalTok{        margByOrder[[}\DecValTok{4}\NormalTok{]]}\SpecialCharTok{$}\NormalTok{list[comb[[}\DecValTok{4}\NormalTok{]]]}
\NormalTok{      )}
\NormalTok{      target.data }\OtherTok{\textless{}{-}} \FunctionTok{c}\NormalTok{(}
\NormalTok{        margByOrder[[}\DecValTok{1}\NormalTok{]]}\SpecialCharTok{$}\NormalTok{marginals[comb[[}\DecValTok{1}\NormalTok{]]],}
\NormalTok{        margByOrder[[}\DecValTok{2}\NormalTok{]]}\SpecialCharTok{$}\NormalTok{marginals[comb[[}\DecValTok{2}\NormalTok{]]],}
\NormalTok{        margByOrder[[}\DecValTok{3}\NormalTok{]]}\SpecialCharTok{$}\NormalTok{marginals[comb[[}\DecValTok{3}\NormalTok{]]],}
\NormalTok{        margByOrder[[}\DecValTok{4}\NormalTok{]]}\SpecialCharTok{$}\NormalTok{marginals[comb[[}\DecValTok{4}\NormalTok{]]]}
\NormalTok{      )}
      
      \FunctionTok{Ipfp}\NormalTok{(}
        \AttributeTok{seed =}\NormalTok{ seed, }
        \AttributeTok{target.list =}\NormalTok{ target.list,}
        \AttributeTok{target.data =}\NormalTok{ target.data,}
        \AttributeTok{iter =} \DecValTok{10000}
\NormalTok{      )}\SpecialCharTok{$}\NormalTok{p.hat}
      
\NormalTok{    \},}
    \AttributeTok{mc.cores =} \DecValTok{256}
\NormalTok{  )}
  
  \DocumentationTok{\#\# return KL Divergence }
  \FunctionTok{sapply}\NormalTok{(}
\NormalTok{    PP, }
    \ControlFlowTok{function}\NormalTok{(p)\{}
      \FunctionTok{sum}\NormalTok{(}\FunctionTok{ifelse}\NormalTok{(f }\SpecialCharTok{\textgreater{}} \DecValTok{0}\NormalTok{, f }\SpecialCharTok{*}\NormalTok{ (}\FunctionTok{log}\NormalTok{(f) }\SpecialCharTok{{-}} \FunctionTok{log}\NormalTok{(p)), }\DecValTok{0}\NormalTok{))}
\NormalTok{    \}}
\NormalTok{  )}
\NormalTok{\}}
\end{Highlighting}
\end{Shaded}

We compute the \textsc{KL} divergence from the empirical \(\prob f\) for
each \(k = 1, \ldots, 6\,893\) sets of marginal constraints for each
subsample \(s\).

To get reproducible results we set a \texttt{seed}

\begin{Shaded}
\begin{Highlighting}[]
\NormalTok{seed }\OtherTok{=} \DecValTok{20220309}
\FunctionTok{set.seed}\NormalTok{(seed)}
\end{Highlighting}
\end{Shaded}

For this analysis we chose a single subsample size \(N_s = N\)

\begin{Shaded}
\begin{Highlighting}[]
\DocumentationTok{\#\# subSampleSize}
\NormalTok{subSampleSize }\OtherTok{=}\NormalTok{ N}
\end{Highlighting}
\end{Shaded}

We determined the \textsc{KL} divergence for each of the \(6\,893\) sets
of marginal constraints in chunks of \texttt{nSamples} \(=100\)
subsamples. The \textsc{KL} divergences for each set of marginal
constraints were saved in a \texttt{RDS} file
\texttt{../data/mixedOrder.\textless{}chunk\ \#\textgreater{}.rds}. In
total we ran 200 chunks resulting in \(20\,000\) subsamples.

\begin{Shaded}
\begin{Highlighting}[]
\DocumentationTok{\#\# number of samples}
\NormalTok{nSamples }\OtherTok{=} \DecValTok{100}

\DocumentationTok{\#\# subsample in chunks of 200 subsamples}
\DocumentationTok{\#\# at a time}
\ControlFlowTok{for}\NormalTok{ (fold }\ControlFlowTok{in} \FunctionTok{seq\_len}\NormalTok{(}\DecValTok{200}\NormalTok{))\{}
  \FunctionTok{cat}\NormalTok{(}\StringTok{"}\SpecialCharTok{\textbackslash{}n}\StringTok{"}\NormalTok{, }\FunctionTok{sprintf}\NormalTok{(}\StringTok{"\%03d"}\NormalTok{, fold), }\StringTok{"}\SpecialCharTok{\textbackslash{}n}\StringTok{"}\NormalTok{)}
\NormalTok{  dkl }\OtherTok{\textless{}{-}} \FunctionTok{lapply}\NormalTok{(}
    \DecValTok{1}\SpecialCharTok{:}\NormalTok{nSamples,}
    \ControlFlowTok{function}\NormalTok{(i)\{}
      \FunctionTok{cat}\NormalTok{(}\StringTok{"."}\NormalTok{)}
      \ControlFlowTok{if}\NormalTok{ (i }\SpecialCharTok{\%\%} \DecValTok{50} \SpecialCharTok{==} \DecValTok{0}\NormalTok{)\{}
        \FunctionTok{cat}\NormalTok{(}\StringTok{" "}\NormalTok{, i, }\StringTok{"}\SpecialCharTok{\textbackslash{}n}\StringTok{"}\NormalTok{)}
\NormalTok{      \}}
      \FunctionTok{subSampleMixed}\NormalTok{(subSampleSize)}
\NormalTok{    \}}
\NormalTok{  )}
  
  \FunctionTok{saveRDS}\NormalTok{(dkl, }\AttributeTok{file =} \FunctionTok{paste0}\NormalTok{(}\StringTok{"../data/mixedOrder."}\NormalTok{, }\FunctionTok{sprintf}\NormalTok{(}\StringTok{"\%03d"}\NormalTok{, fold), }\StringTok{".rds"}\NormalTok{))}
\NormalTok{\}}
\end{Highlighting}
\end{Shaded}

For convenience, we provide a R-script in
\texttt{../R/mixedOrderConstraints.R} that performs the subsampling,
estimation of the models, and calculation of the \textsc{kl} divergence.

\hypertarget{figure-3-1}{%
\paragraph{Figure 3}\label{figure-3-1}}

We calculated the rank of the reduced coefficient matrix \(\mathbf{C}'\)
using the function \texttt{Rank} from the \texttt{pracma} package

\begin{Shaded}
\begin{Highlighting}[]
\ControlFlowTok{if}\NormalTok{ (}\SpecialCharTok{!} \FunctionTok{require}\NormalTok{(}\StringTok{"pracma"}\NormalTok{, }\AttributeTok{quietly =} \ConstantTok{TRUE}\NormalTok{))}
  \FunctionTok{install.packages}\NormalTok{(}\StringTok{"pracma"}\NormalTok{)}
\FunctionTok{library}\NormalTok{(pracma)}
\end{Highlighting}
\end{Shaded}

We loaded the \textsc{KL} divergences for the \(20\,000\) subsamples for
each of the \(6,893\) sets of marginal constraints from the 200 chunks
generated by the R-script in \texttt{../R/mixedOrderConstraints.R}. This
resulted in a matrix \texttt{dklMixed} with \(20,000\) subsamples in the
rows and \(6,893\) sets of marginal constraints in the columns

\begin{Shaded}
\begin{Highlighting}[]
\DocumentationTok{\#\# load DKLs}
\NormalTok{dklMixed }\OtherTok{\textless{}{-}} \FunctionTok{sapply}\NormalTok{(}
    \FunctionTok{seq\_len}\NormalTok{(}\DecValTok{200}\NormalTok{),}
    \ControlFlowTok{function}\NormalTok{(fold)\{}
\NormalTok{        file }\OtherTok{\textless{}{-}} \FunctionTok{paste0}\NormalTok{(}\StringTok{"../data/mixedOrder."}\NormalTok{, }\FunctionTok{sprintf}\NormalTok{(}\StringTok{"\%03d"}\NormalTok{, fold), }\StringTok{".rds"}\NormalTok{)}
        
        \FunctionTok{readRDS}\NormalTok{(file)}
        
\NormalTok{    \}}
\NormalTok{)}
\NormalTok{dklMixed }\OtherTok{\textless{}{-}} \FunctionTok{matrix}\NormalTok{(}\FunctionTok{unlist}\NormalTok{(dklMixed), }\AttributeTok{ncol =} \FunctionTok{length}\NormalTok{(combinations), }\AttributeTok{byrow =} \ConstantTok{TRUE}\NormalTok{)}
\end{Highlighting}
\end{Shaded}

We determined the index of the all first, second, third, and forth order
marginal constraints. For this we generated all combinations using the
function \texttt{getCombinations} in \texttt{../R/getCombinations.R} and
the information about

\begin{Shaded}
\begin{Highlighting}[]
\DocumentationTok{\#\# get the data}
\FunctionTok{source}\NormalTok{(}\StringTok{"../R/getData.R"}\NormalTok{)}
\DocumentationTok{\#\# getCombinations}
\FunctionTok{source}\NormalTok{(}\StringTok{"../R/getCombinations.R"}\NormalTok{)}

\NormalTok{combinations }\OtherTok{\textless{}{-}} \FunctionTok{getCombinations}\NormalTok{(L, L }\SpecialCharTok{{-}} \DecValTok{1}\NormalTok{)}
\NormalTok{nn }\OtherTok{\textless{}{-}}\NormalTok{ combinations}\SpecialCharTok{$}\NormalTok{nn}
\NormalTok{combinations }\OtherTok{\textless{}{-}}\NormalTok{ combinations}\SpecialCharTok{$}\NormalTok{combinations}
\end{Highlighting}
\end{Shaded}

The index is determined by requiring the the combinations contains all 5
first order, 10 second order, 10 third order, and 5 fourth order
constraints:

\begin{Shaded}
\begin{Highlighting}[]
\DocumentationTok{\#\# determine the all ell order constraints}
\NormalTok{allOne }\OtherTok{\textless{}{-}} \FunctionTok{which}\NormalTok{(}\FunctionTok{sapply}\NormalTok{(combinations, }\ControlFlowTok{function}\NormalTok{(x) }\FunctionTok{length}\NormalTok{(x[[}\DecValTok{1}\NormalTok{]]) }\SpecialCharTok{==} \DecValTok{5}\NormalTok{))}
\NormalTok{allTwo }\OtherTok{\textless{}{-}} \FunctionTok{which}\NormalTok{(}\FunctionTok{sapply}\NormalTok{(combinations, }\ControlFlowTok{function}\NormalTok{(x) }\FunctionTok{length}\NormalTok{(x[[}\DecValTok{2}\NormalTok{]]) }\SpecialCharTok{==} \DecValTok{10}\NormalTok{))}
\NormalTok{allThree }\OtherTok{\textless{}{-}} \FunctionTok{which}\NormalTok{(}\FunctionTok{sapply}\NormalTok{(combinations, }\ControlFlowTok{function}\NormalTok{(x) }\FunctionTok{length}\NormalTok{(x[[}\DecValTok{3}\NormalTok{]]) }\SpecialCharTok{==} \DecValTok{10}\NormalTok{))}
\NormalTok{allFour }\OtherTok{\textless{}{-}} \FunctionTok{which}\NormalTok{(}\FunctionTok{sapply}\NormalTok{(combinations, }\ControlFlowTok{function}\NormalTok{(x) }\FunctionTok{length}\NormalTok{(x[[}\DecValTok{4}\NormalTok{]]) }\SpecialCharTok{==} \DecValTok{5}\NormalTok{))}
\end{Highlighting}
\end{Shaded}

We calculate the difference
\(\Delta_{s,k} = \infdiv{\prob f}{\maxentP^{(s,k)}} - \infdiv{\prob f}{\maxentP^{(s,\ell = 4)}}\)
for each of the subsamples \(s =1, \ldots, 20\,000\)

\begin{Shaded}
\begin{Highlighting}[]
\NormalTok{deltas }\OtherTok{\textless{}{-}}\NormalTok{ dklMixed }\SpecialCharTok{{-}}\NormalTok{ dklMixed[, allFour]}
\end{Highlighting}
\end{Shaded}

and the median difference for each set \(k = 1, \ldots, 6\,893\) of
marginal constraints

\begin{Shaded}
\begin{Highlighting}[]
\NormalTok{deltasMedian }\OtherTok{\textless{}{-}} \FunctionTok{apply}\NormalTok{(deltas, }\DecValTok{2}\NormalTok{, median)}
\end{Highlighting}
\end{Shaded}

and order them in ascending order

\begin{Shaded}
\begin{Highlighting}[]
\NormalTok{deltasMedianOrder }\OtherTok{\textless{}{-}} \FunctionTok{order}\NormalTok{(deltasMedian)}
\end{Highlighting}
\end{Shaded}

There are several sets of marginal constraints that give the
\emph{identical} results (see below). Thus we determined the top 10
median \textsc{KL} divergences

\begin{Shaded}
\begin{Highlighting}[]
\NormalTok{top10 }\OtherTok{\textless{}{-}} \FunctionTok{tapply}\NormalTok{(}
\NormalTok{    deltasMedianOrder[}\DecValTok{1}\SpecialCharTok{:}\DecValTok{1000}\NormalTok{],}
    \FunctionTok{round}\NormalTok{(deltasMedian[deltasMedianOrder[}\DecValTok{1}\SpecialCharTok{:}\DecValTok{1000}\NormalTok{]], }\DecValTok{10}\NormalTok{),}
    \ControlFlowTok{function}\NormalTok{(i) i}
\NormalTok{)}
\NormalTok{top10 }\OtherTok{\textless{}{-}} \FunctionTok{sum}\NormalTok{(}\FunctionTok{sapply}\NormalTok{(top10[}\DecValTok{1}\SpecialCharTok{:}\DecValTok{10}\NormalTok{], length))}
\end{Highlighting}
\end{Shaded}

The groups of sets of marginal constraints with \emph{identical} results
can be understood on the level of the linear system, i.e.~they all have
the same reduce row echelon form. We determined the reduced coefficient
matrix \(\mathbf{C}'\) for each set of marginal constraints using the
function \texttt{getCoefficientMatrix} in
\texttt{../R/getCoefficientMatrix.R}

\begin{Shaded}
\begin{Highlighting}[]
\FunctionTok{source}\NormalTok{(}\StringTok{"../R/getCoefficientMatrix.R"}\NormalTok{)}
\end{Highlighting}
\end{Shaded}

The function needs the empirical marginals to determine those marginals
that are zero

\begin{Shaded}
\begin{Highlighting}[]
\DocumentationTok{\#\# marginals}
\FunctionTok{source}\NormalTok{(}\StringTok{"../R/marginalsByOrder.R"}\NormalTok{)}
\NormalTok{margByOrder }\OtherTok{\textless{}{-}} \FunctionTok{lapply}\NormalTok{(}
    \FunctionTok{c}\NormalTok{(}\DecValTok{1}\NormalTok{,}\DecValTok{2}\NormalTok{,}\DecValTok{3}\NormalTok{,}\DecValTok{4}\NormalTok{),}
    \ControlFlowTok{function}\NormalTok{(ell)\{}
        \FunctionTok{marginalsByOrder}\NormalTok{(}
            \AttributeTok{f =}\NormalTok{ f,}
            \AttributeTok{ell =}\NormalTok{ ell,}
            \AttributeTok{L =}\NormalTok{ L}
\NormalTok{        )}
\NormalTok{    \}}
\NormalTok{)}
\end{Highlighting}
\end{Shaded}

Furthermore, it requires the states \(\alpha_i\) for each of the
features

\begin{Shaded}
\begin{Highlighting}[]
\NormalTok{alphas }\OtherTok{\textless{}{-}} \FunctionTok{list}\NormalTok{(}
    \AttributeTok{ageGroup =} \DecValTok{0}\SpecialCharTok{:}\DecValTok{5}\NormalTok{,}
    \AttributeTok{sex =} \DecValTok{0}\SpecialCharTok{:}\DecValTok{1}\NormalTok{,}
    \AttributeTok{arrivalByAmbulance =} \DecValTok{0}\SpecialCharTok{:}\DecValTok{1}\NormalTok{,}
    \AttributeTok{hospitalization =} \DecValTok{0}\SpecialCharTok{:}\DecValTok{1}\NormalTok{,}
    \AttributeTok{critical =} \DecValTok{0}\SpecialCharTok{:}\DecValTok{1}
\NormalTok{)}
\end{Highlighting}
\end{Shaded}

We calculated the reduced coefficient matrix \(\mathbf{C}'\) for each of
the \(6\,893\) sets of constraints

\begin{Shaded}
\begin{Highlighting}[]
\NormalTok{CReduced }\OtherTok{\textless{}{-}} \FunctionTok{lapply}\NormalTok{(}
\NormalTok{    combinations,}
\NormalTok{    getCoefficientMatrix,}
    \AttributeTok{alphas =}\NormalTok{ alphas,}
    \AttributeTok{nn =}\NormalTok{ nn,}
    \AttributeTok{margByOrder =}\NormalTok{ margByOrder}
    
\NormalTok{)}
\end{Highlighting}
\end{Shaded}

Finally, we determined the rank of each reduced coefficient matrix
\(\mathbf{C}'\)

\begin{Shaded}
\begin{Highlighting}[]
\NormalTok{ranks }\OtherTok{\textless{}{-}} \FunctionTok{sapply}\NormalTok{(}
\NormalTok{    CReduced,}
\NormalTok{    Rank}
\NormalTok{)}
\end{Highlighting}
\end{Shaded}

We aggregate the data for the top 10 median \textsc{KL} divergences

\begin{Shaded}
\begin{Highlighting}[]
\NormalTok{dataMixed }\OtherTok{\textless{}{-}} \FunctionTok{list}\NormalTok{(}
    \AttributeTok{deltaDkl =}\NormalTok{ deltas[,deltasMedianOrder[}\FunctionTok{seq\_len}\NormalTok{(top10)]],}
    \AttributeTok{rank =}\NormalTok{ ranks[deltasMedianOrder[}\FunctionTok{seq\_len}\NormalTok{(top10)]],}
    \AttributeTok{combinations =}\NormalTok{ combinations[deltasMedianOrder[}\FunctionTok{seq\_len}\NormalTok{(top10)]],}
    \AttributeTok{coefficientMatrix =}\NormalTok{ CReduced[deltasMedianOrder[}\FunctionTok{seq\_len}\NormalTok{(top10)]],}
    \AttributeTok{allOne =}\NormalTok{ deltasMedian[allOne],}
    \AttributeTok{allTwo =}\NormalTok{ deltasMedian[allTwo],}
    \AttributeTok{allThree =}\NormalTok{ deltasMedian[allThree],}
    \AttributeTok{allFour =}\NormalTok{ deltasMedian[allFour]}
\NormalTok{)}
\end{Highlighting}
\end{Shaded}

and saved the result in the file \texttt{../data/mixedOrder.RData}

\begin{Shaded}
\begin{Highlighting}[]
\FunctionTok{save}\NormalTok{(dataMixed, }\AttributeTok{file =} \StringTok{"../data/mixedOrder.RData"}\NormalTok{)}
\end{Highlighting}
\end{Shaded}

For convenience, we provide a R-script in \texttt{../R/combineMixed.R}
that generates/loads the list \texttt{dataMixed}.

\textbf{Render Figure 4}

Load the data

\begin{Shaded}
\begin{Highlighting}[]
\FunctionTok{source}\NormalTok{(}\StringTok{"../R/combineMixed.R"}\NormalTok{)}
\end{Highlighting}
\end{Shaded}

\begin{Shaded}
\begin{Highlighting}[]
\FunctionTok{pdf}\NormalTok{(}\StringTok{"../figures/Figure4.pdf"}\NormalTok{, }\AttributeTok{width =} \FloatTok{7.08}\NormalTok{, }\AttributeTok{height =} \FloatTok{5.7}\NormalTok{, }\AttributeTok{pointsize =} \DecValTok{8}\NormalTok{)}
\FunctionTok{layout}\NormalTok{(}
  \FunctionTok{matrix}\NormalTok{(}\FunctionTok{c}\NormalTok{(}\DecValTok{1}\NormalTok{, }\DecValTok{2}\NormalTok{, }\DecValTok{3}\NormalTok{, }\DecValTok{3}\NormalTok{), }\AttributeTok{ncol =} \DecValTok{2}\NormalTok{),}
  \AttributeTok{widths =} \FunctionTok{c}\NormalTok{(}\DecValTok{41}\NormalTok{, }\DecValTok{59}\NormalTok{)}
\NormalTok{)}
\end{Highlighting}
\end{Shaded}

Figure 3a:

\begin{Shaded}
\begin{Highlighting}[]
\NormalTok{ylim }\OtherTok{\textless{}{-}} \FunctionTok{range}\NormalTok{(dataMixed}\SpecialCharTok{$}\NormalTok{deltaDkl)}
\NormalTok{medians }\OtherTok{\textless{}{-}} \FunctionTok{apply}\NormalTok{(dataMixed}\SpecialCharTok{$}\NormalTok{deltaDkl, }\DecValTok{2}\NormalTok{, median)}
\NormalTok{tab }\OtherTok{\textless{}{-}} \FunctionTok{table}\NormalTok{(}\FunctionTok{round}\NormalTok{(medians, }\DecValTok{10}\NormalTok{))}
\NormalTok{cs }\OtherTok{\textless{}{-}} \FunctionTok{c}\NormalTok{(}\DecValTok{0}\NormalTok{, }\FunctionTok{cumsum}\NormalTok{(tab))}
\NormalTok{gs }\OtherTok{\textless{}{-}}\NormalTok{ (cs[}\SpecialCharTok{{-}}\DecValTok{1}\NormalTok{] }\SpecialCharTok{{-}}\NormalTok{ cs[}\SpecialCharTok{{-}}\FunctionTok{length}\NormalTok{(cs)]) }\SpecialCharTok{/} \DecValTok{2} \SpecialCharTok{+}\NormalTok{ cs[}\SpecialCharTok{{-}}\FunctionTok{length}\NormalTok{(cs)] }\SpecialCharTok{+} \FloatTok{0.5}
\NormalTok{breaks }\OtherTok{\textless{}{-}} \FunctionTok{seq}\NormalTok{(}\SpecialCharTok{{-}}\FunctionTok{max}\NormalTok{(}\FunctionTok{abs}\NormalTok{(ylim)) }\SpecialCharTok{*} \FloatTok{1.01}\NormalTok{, }\FunctionTok{max}\NormalTok{(}\FunctionTok{abs}\NormalTok{(ylim)) }\SpecialCharTok{*} \FloatTok{1.01}\NormalTok{, }\AttributeTok{length.out =} \DecValTok{102}\NormalTok{)}
\NormalTok{yy }\OtherTok{\textless{}{-}} \FunctionTok{apply}\NormalTok{(dataMixed}\SpecialCharTok{$}\NormalTok{deltaDkl, }\DecValTok{2}\NormalTok{, }\ControlFlowTok{function}\NormalTok{(x)  }\FunctionTok{as.numeric}\NormalTok{(}\FunctionTok{table}\NormalTok{(}\FunctionTok{cut}\NormalTok{(x, breaks))))}

\FunctionTok{plot}\NormalTok{(}\DecValTok{0}\NormalTok{, }\AttributeTok{ylim =}\NormalTok{ ylim, }\AttributeTok{xlim =} \FunctionTok{c}\NormalTok{(}\DecValTok{0}\NormalTok{, }\FunctionTok{ncol}\NormalTok{(yy) }\SpecialCharTok{+} \DecValTok{1}\NormalTok{), }\AttributeTok{type =} \StringTok{"n"}\NormalTok{, }
     \AttributeTok{axes =} \ConstantTok{FALSE}\NormalTok{, }\AttributeTok{ann =} \ConstantTok{FALSE}\NormalTok{, }\AttributeTok{frame =} \ConstantTok{FALSE}
\NormalTok{)}

\DocumentationTok{\#\# violins}
\NormalTok{k }\OtherTok{=} \DecValTok{1}
\ControlFlowTok{for}\NormalTok{ (i }\ControlFlowTok{in} \FunctionTok{seq\_len}\NormalTok{(}\FunctionTok{ncol}\NormalTok{(yy)))\{}
  
  \ControlFlowTok{if}\NormalTok{ (k }\SpecialCharTok{\textless{}} \DecValTok{11}\NormalTok{)\{}
    \ControlFlowTok{if}\NormalTok{ (i }\SpecialCharTok{\textgreater{}}\NormalTok{ gs[k])\{}
\NormalTok{      tmp }\OtherTok{=}\NormalTok{ yy[, i]}
\NormalTok{      tmp }\OtherTok{=}\NormalTok{ (tmp }\SpecialCharTok{/} \FunctionTok{max}\NormalTok{(tmp)) }\SpecialCharTok{*} \DecValTok{3}
\NormalTok{      sel }\OtherTok{=}\NormalTok{ tmp }\SpecialCharTok{\textgreater{}} \DecValTok{0}
\NormalTok{      tmp }\OtherTok{=}\NormalTok{ tmp[tmp }\SpecialCharTok{\textgreater{}} \DecValTok{0}\NormalTok{]}
      \FunctionTok{rect}\NormalTok{(gs[k] }\SpecialCharTok{{-}}\NormalTok{ tmp, breaks[}\SpecialCharTok{{-}}\FunctionTok{length}\NormalTok{(breaks)][sel], }
\NormalTok{           gs[k]}\SpecialCharTok{+}\NormalTok{ tmp, breaks[}\SpecialCharTok{{-}}\DecValTok{1}\NormalTok{][sel], }
           \AttributeTok{col =}\NormalTok{ cols}\SpecialCharTok{$}\NormalTok{grey, }\AttributeTok{border =} \ConstantTok{NA}
\NormalTok{      )}
      
\NormalTok{      k }\OtherTok{\textless{}{-}}\NormalTok{ k }\SpecialCharTok{+} \DecValTok{1}
\NormalTok{    \}}
\NormalTok{  \}}
  
\NormalTok{\}}
\FunctionTok{abline}\NormalTok{(}\AttributeTok{h =}\NormalTok{ dataMixed}\SpecialCharTok{$}\NormalTok{allThree, }\AttributeTok{col =}\NormalTok{ cols}\SpecialCharTok{$}\NormalTok{red)}
\FunctionTok{abline}\NormalTok{(}\AttributeTok{h =}\NormalTok{ dataMixed}\SpecialCharTok{$}\NormalTok{allFour, }\AttributeTok{col =}\NormalTok{ cols}\SpecialCharTok{$}\NormalTok{purple)}

\DocumentationTok{\#\# medians}
\ControlFlowTok{for}\NormalTok{ (i }\ControlFlowTok{in} \FunctionTok{seq\_len}\NormalTok{(}\FunctionTok{ncol}\NormalTok{(yy)))\{}
\NormalTok{  ss }\OtherTok{=} \FunctionTok{round}\NormalTok{(medians[i], }\DecValTok{10}\NormalTok{)}
  \CommentTok{\#points(i {-} 0.3, ss, col = cols$white, cex = 0.5, pch = 16)}
  \ControlFlowTok{if}\NormalTok{ (}\FunctionTok{round}\NormalTok{(ss, }\DecValTok{10}\NormalTok{) }\SpecialCharTok{==} \FunctionTok{round}\NormalTok{(dataMixed}\SpecialCharTok{$}\NormalTok{allThree, }\DecValTok{10}\NormalTok{))\{}
    \FunctionTok{points}\NormalTok{(i, ss, }\AttributeTok{pch =} \DecValTok{16}\NormalTok{, }\AttributeTok{cex =} \FloatTok{0.8}\NormalTok{, }\AttributeTok{col =}\NormalTok{ cols}\SpecialCharTok{$}\NormalTok{red)}
\NormalTok{  \} }\ControlFlowTok{else}\NormalTok{\{}
    \ControlFlowTok{if}\NormalTok{ (}\FunctionTok{round}\NormalTok{(ss, }\DecValTok{10}\NormalTok{) }\SpecialCharTok{==} \FunctionTok{round}\NormalTok{(dataMixed}\SpecialCharTok{$}\NormalTok{allFour, }\DecValTok{10}\NormalTok{))\{}
      \FunctionTok{points}\NormalTok{(i, ss, }\AttributeTok{pch =} \DecValTok{16}\NormalTok{, }\AttributeTok{cex =} \FloatTok{0.8}\NormalTok{, }\AttributeTok{col =}\NormalTok{ cols}\SpecialCharTok{$}\NormalTok{purple)}
\NormalTok{    \} }\ControlFlowTok{else}\NormalTok{\{}
      \FunctionTok{points}\NormalTok{(i, ss, }\AttributeTok{pch =} \DecValTok{16}\NormalTok{, }\AttributeTok{cex =} \FloatTok{0.8}\NormalTok{)}
      
\NormalTok{    \}}
\NormalTok{  \}}
  
\NormalTok{\}}
\FunctionTok{axis}\NormalTok{(}\DecValTok{2}\NormalTok{, }\AttributeTok{at =} \FunctionTok{seq}\NormalTok{(}\SpecialCharTok{{-}}\DecValTok{4}\NormalTok{, }\DecValTok{2}\NormalTok{, }\AttributeTok{by =} \DecValTok{2}\NormalTok{) }\SpecialCharTok{*} \FloatTok{1e{-}5}\NormalTok{, }\AttributeTok{labels =} \FunctionTok{seq}\NormalTok{(}\SpecialCharTok{{-}}\DecValTok{4}\NormalTok{, }\DecValTok{2}\NormalTok{, }\AttributeTok{by =} \DecValTok{2}\NormalTok{), }\AttributeTok{las =} \DecValTok{1}\NormalTok{) }
\FunctionTok{axis}\NormalTok{(}\DecValTok{1}\NormalTok{, }\AttributeTok{at =}\NormalTok{ gs, }\DecValTok{72} \SpecialCharTok{{-}}\NormalTok{ dataMixed}\SpecialCharTok{$}\NormalTok{rank[gs])}

\FunctionTok{mtext}\NormalTok{(tab, }\DecValTok{1}\NormalTok{, }\AttributeTok{line =} \DecValTok{2}\NormalTok{, }\AttributeTok{at =}\NormalTok{ gs)}

\FunctionTok{mtext}\NormalTok{(}\StringTok{"df"}\NormalTok{, }\DecValTok{1}\NormalTok{, }\AttributeTok{line =} \DecValTok{1}\NormalTok{, }\AttributeTok{at =} \SpecialCharTok{{-}}\DecValTok{11}\NormalTok{, }\AttributeTok{adj =} \DecValTok{0}\NormalTok{)}
\FunctionTok{mtext}\NormalTok{(}\StringTok{"\# sets"}\NormalTok{, }\DecValTok{1}\NormalTok{, }\AttributeTok{line =} \DecValTok{2}\NormalTok{, }\AttributeTok{at =} \SpecialCharTok{{-}}\DecValTok{11}\NormalTok{, }\AttributeTok{adj =} \DecValTok{0}\NormalTok{)}
\end{Highlighting}
\end{Shaded}

Figure 3b:

We will need the package \texttt{HyperG}

\begin{Shaded}
\begin{Highlighting}[]
\ControlFlowTok{if}\NormalTok{ (}\SpecialCharTok{!} \FunctionTok{require}\NormalTok{(}\StringTok{"HyperG"}\NormalTok{, }\AttributeTok{quietly =} \ConstantTok{TRUE}\NormalTok{))}
  \FunctionTok{install.packages}\NormalTok{(}\StringTok{"HyperG"}\NormalTok{)}
\FunctionTok{library}\NormalTok{(HyperG)}
\end{Highlighting}
\end{Shaded}

Render the figure

\begin{Shaded}
\begin{Highlighting}[]
\DocumentationTok{\#\# colors for the cardinality of the constraint}
\NormalTok{colors }\OtherTok{=} \FunctionTok{c}\NormalTok{(}\ConstantTok{NA}\NormalTok{, cols}\SpecialCharTok{$}\NormalTok{yellow, cols}\SpecialCharTok{$}\NormalTok{red, cols}\SpecialCharTok{$}\NormalTok{purple)}

\DocumentationTok{\#\# convert the combinations of constraints}
\DocumentationTok{\#\# to a hypergraph in feature space}
\NormalTok{nams }\OtherTok{=} \FunctionTok{c}\NormalTok{(}\StringTok{"age"}\NormalTok{, }\StringTok{"sex"}\NormalTok{, }\StringTok{"amb"}\NormalTok{, }\StringTok{"hosp"}\NormalTok{, }\StringTok{"crit"}\NormalTok{)}
\NormalTok{toHypergraph }\OtherTok{=} \ControlFlowTok{function}\NormalTok{(comb)\{}
\NormalTok{  edges }\OtherTok{=} \FunctionTok{sapply}\NormalTok{(}\DecValTok{1}\SpecialCharTok{:}\DecValTok{4}\NormalTok{, }\ControlFlowTok{function}\NormalTok{(i) }\FunctionTok{matrix}\NormalTok{(nn[[i]]}\SpecialCharTok{$}\NormalTok{c[, comb[[i]]], }\AttributeTok{nrow =}\NormalTok{ i))}
\NormalTok{  edges }\OtherTok{=} \FunctionTok{lapply}\NormalTok{(edges, }\ControlFlowTok{function}\NormalTok{(e) }\FunctionTok{lapply}\NormalTok{(}\FunctionTok{seq\_len}\NormalTok{(}\FunctionTok{ncol}\NormalTok{(e)), }\ControlFlowTok{function}\NormalTok{(i) nams[e[, i]]))}
\NormalTok{  edges }\OtherTok{=} \FunctionTok{unlist}\NormalTok{(edges, }\AttributeTok{recursive =} \ConstantTok{FALSE}\NormalTok{)}
\NormalTok{  cc }\OtherTok{=}\NormalTok{ colors[}\FunctionTok{sapply}\NormalTok{(edges, length)]  }
  \FunctionTok{list}\NormalTok{(}\AttributeTok{gr =} \FunctionTok{hypergraph\_from\_edgelist}\NormalTok{(edges, nams), }\AttributeTok{cols =}\NormalTok{ cc, }\AttributeTok{edges =}\NormalTok{ edges)}
\NormalTok{\}}

\DocumentationTok{\#\# the same coordinates for the five}
\DocumentationTok{\#\# features (a regular pentagon)}
\NormalTok{ang }\OtherTok{=} \DecValTok{2} \SpecialCharTok{*}\NormalTok{ pi }\SpecialCharTok{/} \DecValTok{5} \SpecialCharTok{*} \DecValTok{1}\SpecialCharTok{:}\DecValTok{5}
\NormalTok{LO }\OtherTok{=} \FunctionTok{cbind}\NormalTok{(}
  \FunctionTok{sin}\NormalTok{(ang) }\SpecialCharTok{*} \FloatTok{0.8}\NormalTok{,}
  \FunctionTok{cos}\NormalTok{(ang) }\SpecialCharTok{*} \FloatTok{0.8}
\NormalTok{)}

\DocumentationTok{\#\# there are 13 sets of constraints}
\DocumentationTok{\#\# with the same reduced row echelon form}
\DocumentationTok{\#\# we define the offsets on the plot}
\NormalTok{offs }\OtherTok{=} \FunctionTok{cbind}\NormalTok{(}
  \FunctionTok{c}\NormalTok{(}\DecValTok{2}\NormalTok{,}\DecValTok{4}\NormalTok{,}\DecValTok{6}\NormalTok{, }\DecValTok{1}\NormalTok{,}\DecValTok{3}\NormalTok{,}\DecValTok{5}\NormalTok{,}\DecValTok{7}\NormalTok{,}\DecValTok{1}\NormalTok{,}\DecValTok{3}\NormalTok{,}\DecValTok{5}\NormalTok{,}\DecValTok{7}\NormalTok{, }\DecValTok{3}\NormalTok{,}\DecValTok{5}\NormalTok{),}
  \FunctionTok{c}\NormalTok{(}\FunctionTok{rep}\NormalTok{(}\DecValTok{7}\NormalTok{, }\DecValTok{3}\NormalTok{), }\FunctionTok{rep}\NormalTok{(}\DecValTok{5}\NormalTok{, }\DecValTok{4}\NormalTok{), }\FunctionTok{rep}\NormalTok{(}\DecValTok{3}\NormalTok{,}\DecValTok{4}\NormalTok{), }\FunctionTok{rep}\NormalTok{(}\DecValTok{1}\NormalTok{, }\DecValTok{2}\NormalTok{))}
\NormalTok{)}

\DocumentationTok{\#\# render the graphs}
\FunctionTok{par}\NormalTok{(}\AttributeTok{mar =} \FunctionTok{c}\NormalTok{(}\DecValTok{0}\NormalTok{,}\DecValTok{0}\NormalTok{,}\DecValTok{3}\NormalTok{,}\DecValTok{3}\NormalTok{) }\SpecialCharTok{+} \FloatTok{0.1}\NormalTok{)}
\ControlFlowTok{for}\NormalTok{ (i }\ControlFlowTok{in} \DecValTok{1}\SpecialCharTok{:}\DecValTok{13}\NormalTok{)\{}
\NormalTok{  gg }\OtherTok{=} \FunctionTok{toHypergraph}\NormalTok{(dataMixed}\SpecialCharTok{$}\NormalTok{combinations[[i]])}
\NormalTok{  LOtmp }\OtherTok{=}\NormalTok{ LO}
\NormalTok{  LOtmp[,}\DecValTok{1}\NormalTok{] }\OtherTok{=}\NormalTok{ LOtmp[,}\DecValTok{1}\NormalTok{] }\SpecialCharTok{+}\NormalTok{ offs[i,}\DecValTok{1}\NormalTok{]}
\NormalTok{  LOtmp[,}\DecValTok{2}\NormalTok{] }\OtherTok{=}\NormalTok{ LOtmp[,}\DecValTok{2}\NormalTok{] }\SpecialCharTok{+}\NormalTok{ offs[i,}\DecValTok{2}\NormalTok{]}
  \FunctionTok{plot.hypergraph}\NormalTok{(gg}\SpecialCharTok{$}\NormalTok{gr, }\AttributeTok{mark.col =} \FunctionTok{adjustcolor}\NormalTok{(gg}\SpecialCharTok{$}\NormalTok{cols, }\FloatTok{0.3}\NormalTok{), }\AttributeTok{mark.border =}\NormalTok{ gg}\SpecialCharTok{$}\NormalTok{cols, }
                  \AttributeTok{vertex.size =} \DecValTok{30}\NormalTok{, }\AttributeTok{vertex.label.cex =} \DecValTok{1}\NormalTok{, }\AttributeTok{layout =}\NormalTok{ LOtmp, }
                  \AttributeTok{xlim =} \FunctionTok{c}\NormalTok{(}\DecValTok{0}\NormalTok{, }\DecValTok{8}\NormalTok{), }\AttributeTok{ylim =} \FunctionTok{c}\NormalTok{(}\DecValTok{0}\NormalTok{,}\DecValTok{8}\NormalTok{), }\AttributeTok{add =}\NormalTok{ i }\SpecialCharTok{!=} \DecValTok{1}\NormalTok{, }\AttributeTok{rescale =} \ConstantTok{FALSE}\NormalTok{, }
                  \AttributeTok{vertex.color =} \FunctionTok{c}\NormalTok{(cols}\SpecialCharTok{$}\NormalTok{white, cols}\SpecialCharTok{$}\NormalTok{black, cols}\SpecialCharTok{$}\NormalTok{yellow, cols}\SpecialCharTok{$}\NormalTok{lightblue, cols}\SpecialCharTok{$}\NormalTok{buff), }
                  \AttributeTok{vertex.label =} \ConstantTok{NA}
\NormalTok{  )}
  
\NormalTok{\}}
\DocumentationTok{\#\# add the legend}
\FunctionTok{par}\NormalTok{(}\AttributeTok{xpd =} \ConstantTok{TRUE}\NormalTok{)}
\FunctionTok{points}\NormalTok{(}\FunctionTok{c}\NormalTok{(}\DecValTok{0}\NormalTok{, }\DecValTok{2}\NormalTok{, }\DecValTok{3}\NormalTok{,}\DecValTok{5}\NormalTok{,}\FloatTok{7.5}\NormalTok{), }
       \FunctionTok{rep}\NormalTok{(}\DecValTok{9}\NormalTok{, }\DecValTok{5}\NormalTok{), }\AttributeTok{pch =} \DecValTok{21}\NormalTok{, }
       \AttributeTok{bg =} \FunctionTok{c}\NormalTok{(cols}\SpecialCharTok{$}\NormalTok{white, cols}\SpecialCharTok{$}\NormalTok{black, cols}\SpecialCharTok{$}\NormalTok{yellow, cols}\SpecialCharTok{$}\NormalTok{lightblue, cols}\SpecialCharTok{$}\NormalTok{buff), }
       \AttributeTok{cex =} \FloatTok{1.5}
\NormalTok{)}
\FunctionTok{text}\NormalTok{(}\FunctionTok{c}\NormalTok{(}\DecValTok{0}\NormalTok{, }\DecValTok{2}\NormalTok{, }\DecValTok{3}\NormalTok{,}\DecValTok{5}\NormalTok{,}\FloatTok{7.5}\NormalTok{), }\FunctionTok{rep}\NormalTok{(}\DecValTok{9}\NormalTok{, }\DecValTok{5}\NormalTok{), }
     \AttributeTok{labels =} \FunctionTok{c}\NormalTok{(}\StringTok{"age group"}\NormalTok{, }\StringTok{"sex"}\NormalTok{, }\StringTok{"ambulance"}\NormalTok{, }\StringTok{"hospitalization"}\NormalTok{, }\StringTok{"critical"}\NormalTok{), }
     \AttributeTok{pos =} \DecValTok{4}
\NormalTok{)}
\FunctionTok{par}\NormalTok{(}\AttributeTok{xpd =} \ConstantTok{FALSE}\NormalTok{)}
\end{Highlighting}
\end{Shaded}

Figure 3c:

Compute the reduced row echelon form \(\mathbf{R}\) and the
transformation matrix \(\mathbf{T}\) from the reduced coefficient matrix
\(\mathbf{C}'\) for the top scoring set of constraints

\begin{Shaded}
\begin{Highlighting}[]
\NormalTok{i }\OtherTok{\textless{}{-}} \DecValTok{1}
\NormalTok{T }\OtherTok{\textless{}{-}} \FunctionTok{rref}\NormalTok{(}
  \FunctionTok{cbind}\NormalTok{(}
\NormalTok{    dataMixed}\SpecialCharTok{$}\NormalTok{coefficientMatrix[[i]], }
    \FunctionTok{diag}\NormalTok{(}\FunctionTok{nrow}\NormalTok{(dataMixed}\SpecialCharTok{$}\NormalTok{coefficientMatrix[[i]]))}
\NormalTok{  )}
\NormalTok{)}
\NormalTok{R }\OtherTok{\textless{}{-}}\NormalTok{ T[, }\DecValTok{1}\SpecialCharTok{:}\DecValTok{72}\NormalTok{]}
\NormalTok{T }\OtherTok{\textless{}{-}}\NormalTok{ T[, }\SpecialCharTok{{-}}\NormalTok{(}\DecValTok{1}\SpecialCharTok{:}\DecValTok{72}\NormalTok{)]}
\end{Highlighting}
\end{Shaded}

Get the marginal states and the number of rows

\begin{Shaded}
\begin{Highlighting}[]
\FunctionTok{source}\NormalTok{(}\StringTok{"../R/getMarginals.R"}\NormalTok{)}
\NormalTok{m }\OtherTok{\textless{}{-}} \FunctionTok{getMarginals}\NormalTok{(dataMixed}\SpecialCharTok{$}\NormalTok{combinations[[i]], alphas, nn, margByOrder)}
\NormalTok{marginalStates }\OtherTok{\textless{}{-}} \FunctionTok{getMarginalsState}\NormalTok{(dataMixed}\SpecialCharTok{$}\NormalTok{combinations[[i]], alphas, nn, margByOrder, L)}
\NormalTok{rows }\OtherTok{\textless{}{-}}\NormalTok{ marginalStates}\SpecialCharTok{$}\NormalTok{rows}
\NormalTok{marginalStates }\OtherTok{\textless{}{-}}\NormalTok{ marginalStates}\SpecialCharTok{$}\NormalTok{m}
\end{Highlighting}
\end{Shaded}

Select nonzero rows in \(\mathbf{R}\)

\begin{Shaded}
\begin{Highlighting}[]
\NormalTok{sel }\OtherTok{\textless{}{-}} \FunctionTok{which}\NormalTok{(}\FunctionTok{rowSums}\NormalTok{(R }\SpecialCharTok{!=} \DecValTok{0}\NormalTok{) }\SpecialCharTok{\textgreater{}} \DecValTok{0}\NormalTok{)}
\NormalTok{R }\OtherTok{\textless{}{-}}\NormalTok{ R[sel, ]}
\end{Highlighting}
\end{Shaded}

All admissible microstates

\begin{Shaded}
\begin{Highlighting}[]
\NormalTok{microstates }\OtherTok{\textless{}{-}}\NormalTok{ empirical[, }\DecValTok{1}\SpecialCharTok{:}\DecValTok{5}\NormalTok{][empirical}\SpecialCharTok{$}\NormalTok{count }\SpecialCharTok{!=} \DecValTok{0}\NormalTok{, ]}
\end{Highlighting}
\end{Shaded}

Render the figure

\begin{Shaded}
\begin{Highlighting}[]
\FunctionTok{par}\NormalTok{(}\AttributeTok{mar =} \FunctionTok{c}\NormalTok{(}\DecValTok{0}\NormalTok{,}\DecValTok{0}\NormalTok{,}\DecValTok{0}\NormalTok{,}\DecValTok{0}\NormalTok{) }\SpecialCharTok{+} \FloatTok{0.1}\NormalTok{)}
\FunctionTok{plot}\NormalTok{(}\ConstantTok{NULL}\NormalTok{, }\AttributeTok{ylim =} \FunctionTok{c}\NormalTok{(}\DecValTok{0}\NormalTok{, }\FunctionTok{nrow}\NormalTok{(T) }\SpecialCharTok{+} \DecValTok{2}\NormalTok{), }\AttributeTok{xlim =} \FunctionTok{c}\NormalTok{(}\DecValTok{0}\NormalTok{,}\FunctionTok{nrow}\NormalTok{(T) }\SpecialCharTok{+} \DecValTok{1}\NormalTok{), }
     \AttributeTok{type =} \StringTok{"n"}\NormalTok{, }\AttributeTok{frame =} \ConstantTok{FALSE}\NormalTok{, }\AttributeTok{axes =} \ConstantTok{FALSE}\NormalTok{, }\AttributeTok{ann =} \ConstantTok{FALSE}\NormalTok{)}

\DocumentationTok{\#\# marginal constraints}
\NormalTok{cr }\OtherTok{\textless{}{-}} \FunctionTok{colorRampPalette}\NormalTok{(}\FunctionTok{c}\NormalTok{(cols}\SpecialCharTok{$}\NormalTok{white, cols}\SpecialCharTok{$}\NormalTok{black))(}\DecValTok{6}\NormalTok{)}
\NormalTok{yy }\OtherTok{\textless{}{-}} \DecValTok{1}\SpecialCharTok{:}\FunctionTok{nrow}\NormalTok{(marginalStates)}
\NormalTok{ss }\OtherTok{\textless{}{-}} \FunctionTok{which}\NormalTok{(}\SpecialCharTok{!}\FunctionTok{is.na}\NormalTok{(marginalStates[, }\DecValTok{1}\NormalTok{]))}
\FunctionTok{points}\NormalTok{(}
  \DecValTok{10}\SpecialCharTok{:}\DecValTok{14}\NormalTok{, }\FunctionTok{rep}\NormalTok{(}\FloatTok{100.5}\NormalTok{,}\DecValTok{5}\NormalTok{), }\AttributeTok{pch =} \DecValTok{21}\NormalTok{, }
  \AttributeTok{bg =} \FunctionTok{c}\NormalTok{(cols}\SpecialCharTok{$}\NormalTok{white, cols}\SpecialCharTok{$}\NormalTok{black, cols}\SpecialCharTok{$}\NormalTok{yellow, cols}\SpecialCharTok{$}\NormalTok{lightblue, cols}\SpecialCharTok{$}\NormalTok{buff), }
  \AttributeTok{cex =} \FloatTok{0.7}\NormalTok{, }\AttributeTok{lwd =} \FloatTok{0.5}
\NormalTok{)}
\FunctionTok{rect}\NormalTok{(}
  \DecValTok{10} \SpecialCharTok{{-}} \FloatTok{0.5}\NormalTok{, yy[ss] }\SpecialCharTok{{-}} \FloatTok{0.5}\NormalTok{, }\DecValTok{10} \SpecialCharTok{+} \FloatTok{0.5}\NormalTok{, yy[ss] }\SpecialCharTok{+} \FloatTok{0.5}\NormalTok{, }
  \AttributeTok{col =}\NormalTok{ cr[marginalStates[ss, }\DecValTok{1}\NormalTok{] }\SpecialCharTok{+} \DecValTok{1}\NormalTok{], }\AttributeTok{border =}\NormalTok{ cols}\SpecialCharTok{$}\NormalTok{grey, }
  \AttributeTok{lwd =} \FloatTok{0.2}
\NormalTok{)}

\NormalTok{cr }\OtherTok{\textless{}{-}} \FunctionTok{c}\NormalTok{(cols}\SpecialCharTok{$}\NormalTok{white, cols}\SpecialCharTok{$}\NormalTok{black)}
\ControlFlowTok{for}\NormalTok{ (x }\ControlFlowTok{in} \DecValTok{2}\SpecialCharTok{:}\DecValTok{5}\NormalTok{)\{}
\NormalTok{  ss }\OtherTok{\textless{}{-}} \FunctionTok{which}\NormalTok{(}\SpecialCharTok{!}\FunctionTok{is.na}\NormalTok{(marginalStates[, x]))}
  \FunctionTok{rect}\NormalTok{(}
    \DecValTok{9} \SpecialCharTok{+}\NormalTok{ x }\SpecialCharTok{{-}} \FloatTok{0.5}\NormalTok{, yy[ss] }\SpecialCharTok{{-}} \FloatTok{0.5}\NormalTok{, }\DecValTok{9}\SpecialCharTok{+}\NormalTok{x }\SpecialCharTok{+} \FloatTok{0.5}\NormalTok{, yy[ss] }\SpecialCharTok{+} \FloatTok{0.5}\NormalTok{, }
    \AttributeTok{col =}\NormalTok{ cr[marginalStates[ss, x] }\SpecialCharTok{+} \DecValTok{1}\NormalTok{], }\AttributeTok{border =}\NormalTok{ cols}\SpecialCharTok{$}\NormalTok{grey, }
    \AttributeTok{lwd =} \FloatTok{0.2}
\NormalTok{  )}
\NormalTok{\}}

\DocumentationTok{\#\# labels for the sets of constraints}
\NormalTok{y1 }\OtherTok{\textless{}{-}} \FunctionTok{c}\NormalTok{(}\DecValTok{0}\NormalTok{, }\FunctionTok{cumsum}\NormalTok{(rows)[}\SpecialCharTok{{-}}\FunctionTok{length}\NormalTok{(rows)]) }\SpecialCharTok{+} \FloatTok{0.5}
\NormalTok{y2 }\OtherTok{\textless{}{-}} \FunctionTok{cumsum}\NormalTok{(rows) }\SpecialCharTok{+} \FloatTok{0.5}

\FunctionTok{rect}\NormalTok{(}
  \FloatTok{8.5} \SpecialCharTok{{-}} \FloatTok{0.25} \SpecialCharTok{*}\NormalTok{ (}\FunctionTok{seq\_along}\NormalTok{(y1) }\SpecialCharTok{{-}} \DecValTok{1}\NormalTok{), y1, }\DecValTok{9} \SpecialCharTok{{-}} \FloatTok{0.25} \SpecialCharTok{*}\NormalTok{ (}\FunctionTok{seq\_along}\NormalTok{(y1) }\SpecialCharTok{{-}} \DecValTok{1}\NormalTok{), y2, }
  \AttributeTok{col =}\NormalTok{ cols}\SpecialCharTok{$}\NormalTok{purple, }\AttributeTok{border =} \ConstantTok{NA}
\NormalTok{)}
\NormalTok{k }\OtherTok{\textless{}{-}} \DecValTok{0}
\NormalTok{cr }\OtherTok{\textless{}{-}} \FunctionTok{c}\NormalTok{(cols}\SpecialCharTok{$}\NormalTok{white, cols}\SpecialCharTok{$}\NormalTok{black, cols}\SpecialCharTok{$}\NormalTok{yellow, cols}\SpecialCharTok{$}\NormalTok{lightblue, cols}\SpecialCharTok{$}\NormalTok{buff)}
\NormalTok{marginalLabel }\OtherTok{\textless{}{-}}
  \FunctionTok{lapply}\NormalTok{(}
    \DecValTok{1}\SpecialCharTok{:}\FunctionTok{length}\NormalTok{(dataMixed}\SpecialCharTok{$}\NormalTok{combinations[[i]]),}
    \ControlFlowTok{function}\NormalTok{(ell)\{}
\NormalTok{      m }\OtherTok{\textless{}{-}} \ConstantTok{NULL}
      \ControlFlowTok{for}\NormalTok{(j }\ControlFlowTok{in}\NormalTok{ dataMixed}\SpecialCharTok{$}\NormalTok{combinations[[i]][[ell]])\{}
\NormalTok{        m }\OtherTok{\textless{}{-}} \FunctionTok{cbind}\NormalTok{(m, nn[[ell]]}\SpecialCharTok{$}\NormalTok{c[, j])}
        \FunctionTok{points}\NormalTok{(}
          \FloatTok{8.5} \SpecialCharTok{{-}} \FloatTok{0.25} \SpecialCharTok{*}\NormalTok{ k }\SpecialCharTok{{-}}\NormalTok{ (}\DecValTok{1}\SpecialCharTok{:}\NormalTok{ell) }\SpecialCharTok{*} \FloatTok{1.5}\NormalTok{, }\FunctionTok{rep}\NormalTok{((y1 }\SpecialCharTok{+}\NormalTok{ y2)[k }\SpecialCharTok{+} \DecValTok{1}\NormalTok{] }\SpecialCharTok{/} \DecValTok{2}\NormalTok{, ell), }
          \AttributeTok{bg =}\NormalTok{ cr[}\FunctionTok{rev}\NormalTok{(nn[[ell]]}\SpecialCharTok{$}\NormalTok{c[, j])], }\AttributeTok{pch =} \DecValTok{21}\NormalTok{, }\AttributeTok{lwd =} \FloatTok{0.5}
\NormalTok{        )}
\NormalTok{        k }\OtherTok{\textless{}\textless{}{-}}\NormalTok{ k }\SpecialCharTok{+} \DecValTok{1}   
\NormalTok{      \}}
\NormalTok{      m}
\NormalTok{    \}}
\NormalTok{  )}

\DocumentationTok{\#\# links between marginals and generalized marginal constraints}
\NormalTok{rr1 }\OtherTok{\textless{}{-}} \FunctionTok{which}\NormalTok{(T }\SpecialCharTok{!=} \DecValTok{0}\NormalTok{, }\AttributeTok{arr.ind =} \ConstantTok{TRUE}\NormalTok{)}
\FunctionTok{segments}\NormalTok{(}
  \DecValTok{15} \SpecialCharTok{+} \FloatTok{0.5}\NormalTok{, rr1[,}\DecValTok{2}\NormalTok{], }\DecValTok{55} \SpecialCharTok{{-}} \FloatTok{0.5}\NormalTok{, rr1[,}\DecValTok{1}\NormalTok{], }
  \AttributeTok{col =} \FunctionTok{ifelse}\NormalTok{(T[rr1] }\SpecialCharTok{\textgreater{}} \DecValTok{0}\NormalTok{, cols}\SpecialCharTok{$}\NormalTok{red, cols}\SpecialCharTok{$}\NormalTok{blue),}
  \AttributeTok{lwd =} \FloatTok{0.2}
\NormalTok{)}

\DocumentationTok{\#\# generalized marginal constraints}
\NormalTok{yy2 }\OtherTok{\textless{}{-}} \DecValTok{1}\SpecialCharTok{:}\FunctionTok{length}\NormalTok{(sel)}
\FunctionTok{rect}\NormalTok{(}
  \DecValTok{56} \SpecialCharTok{{-}} \FloatTok{0.5}\NormalTok{, yy2 }\SpecialCharTok{{-}} \FloatTok{0.5}\NormalTok{, }\DecValTok{56} \SpecialCharTok{+} \FloatTok{0.5}\NormalTok{, yy2 }\SpecialCharTok{+} \FloatTok{0.5}\NormalTok{, }
  \AttributeTok{col =}\NormalTok{ cols}\SpecialCharTok{$}\NormalTok{black, }\AttributeTok{border =}\NormalTok{ cols}\SpecialCharTok{$}\NormalTok{grey,}
  \AttributeTok{lwd =} \FloatTok{0.2}
\NormalTok{)}
\NormalTok{yy2 }\OtherTok{\textless{}{-}}\NormalTok{ (}\FunctionTok{length}\NormalTok{(sel) }\SpecialCharTok{+} \DecValTok{1}\NormalTok{)}\SpecialCharTok{:}\FunctionTok{nrow}\NormalTok{(T)}
\FunctionTok{rect}\NormalTok{(}
  \DecValTok{56} \SpecialCharTok{{-}} \FloatTok{0.5}\NormalTok{, yy2 }\SpecialCharTok{{-}} \FloatTok{0.5}\NormalTok{, }\DecValTok{56} \SpecialCharTok{+} \FloatTok{0.5}\NormalTok{, yy2 }\SpecialCharTok{+} \FloatTok{0.5}\NormalTok{, }
  \AttributeTok{col =}\NormalTok{ cols}\SpecialCharTok{$}\NormalTok{white, }\AttributeTok{border =}\NormalTok{ cols}\SpecialCharTok{$}\NormalTok{grey,}
  \AttributeTok{lwd =} \FloatTok{0.2}
\NormalTok{)}

\DocumentationTok{\#\# links between generalized marginal constraints and joint distribution}
\NormalTok{rr1 }\OtherTok{\textless{}{-}} \FunctionTok{which}\NormalTok{(R }\SpecialCharTok{!=} \DecValTok{0}\NormalTok{, }\AttributeTok{arr.ind =} \ConstantTok{TRUE}\NormalTok{)}
\FunctionTok{segments}\NormalTok{(}
  \DecValTok{57} \SpecialCharTok{+} \FloatTok{0.5}\NormalTok{, rr1[,}\DecValTok{1}\NormalTok{], }\DecValTok{96} \SpecialCharTok{+} \FloatTok{0.5}\NormalTok{, rr1[,}\DecValTok{2}\NormalTok{], }
  \AttributeTok{col =} \FunctionTok{ifelse}\NormalTok{(R[rr1] }\SpecialCharTok{\textgreater{}} \DecValTok{0}\NormalTok{, cols}\SpecialCharTok{$}\NormalTok{red, cols}\SpecialCharTok{$}\NormalTok{blue),}
  \AttributeTok{lwd =} \FloatTok{0.2}
\NormalTok{)}

\DocumentationTok{\#\# joint distribution}
\NormalTok{cr }\OtherTok{\textless{}{-}} \FunctionTok{colorRampPalette}\NormalTok{(}\FunctionTok{c}\NormalTok{(cols}\SpecialCharTok{$}\NormalTok{white, cols}\SpecialCharTok{$}\NormalTok{black))(}\DecValTok{6}\NormalTok{)}
\NormalTok{yy }\OtherTok{\textless{}{-}} \DecValTok{1}\SpecialCharTok{:}\DecValTok{72}
\FunctionTok{rect}\NormalTok{(}
  \DecValTok{98} \SpecialCharTok{{-}} \FloatTok{0.5}\NormalTok{, yy }\SpecialCharTok{{-}} \FloatTok{0.5}\NormalTok{, }\DecValTok{98} \SpecialCharTok{+} \FloatTok{0.5}\NormalTok{, yy }\SpecialCharTok{+} \FloatTok{0.5}\NormalTok{, }
  \AttributeTok{col =}\NormalTok{ cr[microstates[, }\DecValTok{1}\NormalTok{] }\SpecialCharTok{+} \DecValTok{1}\NormalTok{], }\AttributeTok{border =}\NormalTok{ cols}\SpecialCharTok{$}\NormalTok{grey,}
  \AttributeTok{lwd =} \FloatTok{0.2}
\NormalTok{)}
\NormalTok{cr }\OtherTok{\textless{}{-}} \FunctionTok{c}\NormalTok{(cols}\SpecialCharTok{$}\NormalTok{white, cols}\SpecialCharTok{$}\NormalTok{black)}
\ControlFlowTok{for}\NormalTok{ (x }\ControlFlowTok{in} \DecValTok{2}\SpecialCharTok{:}\DecValTok{5}\NormalTok{)}
  \FunctionTok{rect}\NormalTok{(}
    \DecValTok{97} \SpecialCharTok{+}\NormalTok{ x }\SpecialCharTok{{-}} \FloatTok{0.5}\NormalTok{, yy }\SpecialCharTok{{-}} \FloatTok{0.5}\NormalTok{, }\DecValTok{97} \SpecialCharTok{+}\NormalTok{ x }\SpecialCharTok{+} \FloatTok{0.5}\NormalTok{, yy }\SpecialCharTok{+} \FloatTok{0.5}\NormalTok{, }
    \AttributeTok{col =}\NormalTok{ cr[microstates[, x] }\SpecialCharTok{+} \DecValTok{1}\NormalTok{], }\AttributeTok{border =}\NormalTok{ cols}\SpecialCharTok{$}\NormalTok{grey,}
    \AttributeTok{lwd =} \FloatTok{0.2}
\NormalTok{  )}

\FunctionTok{points}\NormalTok{(}
  \DecValTok{98}\SpecialCharTok{:}\DecValTok{102}\NormalTok{, }\FunctionTok{rep}\NormalTok{(}\FloatTok{73.5}\NormalTok{,}\DecValTok{5}\NormalTok{), }\AttributeTok{pch =} \DecValTok{21}\NormalTok{, }
  \AttributeTok{bg =} \FunctionTok{c}\NormalTok{(cols}\SpecialCharTok{$}\NormalTok{white, cols}\SpecialCharTok{$}\NormalTok{black, cols}\SpecialCharTok{$}\NormalTok{yellow, cols}\SpecialCharTok{$}\NormalTok{lightblue, cols}\SpecialCharTok{$}\NormalTok{buff), }
  \AttributeTok{cex =} \FloatTok{0.7}\NormalTok{, }\AttributeTok{lwd =} \FloatTok{0.5}
\NormalTok{)}

\DocumentationTok{\#\# legend}
\FunctionTok{segments}\NormalTok{(}\DecValTok{95}\NormalTok{, }\DecValTok{99}\NormalTok{, }\DecValTok{97}\NormalTok{, }\DecValTok{99}\NormalTok{, }\AttributeTok{col =}\NormalTok{ cols}\SpecialCharTok{$}\NormalTok{red)}
\FunctionTok{segments}\NormalTok{(}\DecValTok{95}\NormalTok{, }\DecValTok{96}\NormalTok{, }\DecValTok{97}\NormalTok{, }\DecValTok{96}\NormalTok{, }\AttributeTok{col =}\NormalTok{ cols}\SpecialCharTok{$}\NormalTok{blue)}
\FunctionTok{text}\NormalTok{(}\DecValTok{97}\NormalTok{,}\DecValTok{99}\NormalTok{, }\StringTok{"+1"}\NormalTok{, }\AttributeTok{pos =} \DecValTok{4}\NormalTok{)}
\FunctionTok{text}\NormalTok{(}\DecValTok{97}\NormalTok{,}\DecValTok{96}\NormalTok{, }\StringTok{"{-}1"}\NormalTok{, }\AttributeTok{pos =} \DecValTok{4}\NormalTok{)}
\end{Highlighting}
\end{Shaded}


\end{document}
